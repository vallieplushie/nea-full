% Copyright 2018-2022 by Romano Giannetti
% Copyright 2015-2022 by Stefan Lindner
% Copyright 2013-2022 by Stefan Erhardt
% Copyright 2007-2022 by Massimo Redaelli
%
% This file may be distributed and/or modified
%
% 1. under the LaTeX Project Public License and/or
% 2. under the GNU Public License.
%
% See the files gpl-3.0_license.txt and lppl-1-3c_license.txt for more details.

\def\pgf@circ@handleSI#1{
    \noexpandarg
    \def\pgf@temp{}
    \StrBetween{#1}{<}{>}[\pgf@circ@handleSI@unit]
    \StrLen{\pgf@circ@handleSI@unit}[\pgf@circ@handleSI@unit@len]

    \ifnum\pgf@circ@handleSI@unit@len=0
    \pgf@circ@siunitx@resfalse
    \else
    \IfEndWith{#1}{>}{
        \pgf@circ@siunitx@restrue
        \noexpandarg
        \StrBefore{#1}{<}[\pgf@circ@handleSI@val]
        %\typeout{si |#1|}
        }{
        \pgf@circ@siunitx@resfalse
        %\typeout{no si |#1|}
    }
\fi
}

\def\pgf@circ@ifkeyempty#1{
    \pgfextra{
        \ctikzset{#1/.get=\pgf@circ@temp}
        \edef\pgf@temp{}
    }
    \ifx\pgf@circ@temp\pgf@temp
    }

%%%%%%%%%%%%%%%%%%%%%%%%%%%%%%%%
%%    Math routines

\def\pgf@circ@stripdecimals#1.#2\pgf@nil{#1}

%%%%%%%%%%%%%%%%%%%%%%%%%%%%%
%% useful commands

\ifpgfutil@format@is@latex
    %% flipping text
    \def\ctikzflipx#1{\scalebox{-1}[1]{#1}}
    \def\ctikzflipy#1{\scalebox{1}[-1]{#1}}
    \def\ctikzflipxy#1{\scalebox{-1}[-1]{#1}}
    % text mode overbar
    % Thanks to @egreg https://tex.stackexchange.com/a/24133/38080
    \def\ctikztextnot#1{$\overline{\hbox{#1}}\m@th$}
\else\ifpgfutil@format@is@plain
    % text mode overbar
    % but really circuitikz will not work in plain...
    % Thanks to @egreg https://tex.stackexchange.com/a/24133/38080
    \def\ctikztextnot#1{$\overline{\hbox{#1}}$}
\else\ifpgfutil@format@is@context
    %% flipping text
    \def\ctikzflipx#1{\mirror{#1}}
    \def\ctikzflipy#1{\mirror{\rotate[rotation=180]{#1}}}
    \def\ctikzflipxy#1{\rotate[rotation=180]{#1}}
    % text mode overbar
    % Thanks to @egreg https://tex.stackexchange.com/a/24133/38080
    \def\ctikztextnot#1{$\overline{\hbox{#1}}$}
\fi\fi\fi

%%%%%%%%%%%%%%%%%%%%%%%%%%%
%% switch to use fpu in reciprocal scale transformations
%%
%% this code has been contributed by Schrödinger's cat
%% https://tex.stackexchange.com/a/529159/38080
%%
%% Use the official key to use the fpu if installed, see
%% https://github.com/pgf-tikz/pgf/issues/861
%%
%% Thanks to "muzimuzhi Z" https://tex.stackexchange.com/a/547085/38080
%% Thanks to Henri Menke for a faster approach https://github.com/circuitikz/circuitikz/commit/00966c45c42b464fab5429f89f2b7fb414e9b3f7#commitcomment-54592494
%%
\pgfkeysifdefined{/pgf/fpu/install only/.@cmd}{%
    \pgfqkeys{/pgf}{use fpu reciprocal/.code={\pgfkeys{/pgf/fpu/install only={reciprocal}}}}%
    }{%
    \pgfkeysifdefined{/pgf/use fpu reciprocal/.@cmd}{% use stock one
        }{
        \pgfqkeys{/pgf}{use fpu reciprocal/.code={%
                \def\pgfmathreciprocal@##1{%
                    \begingroup
                    \pgfmathfloatparsenumber{##1}%
                    \pgfmathfloatreciprocal@{\pgfmathresult}%
                    \pgfmathfloattofixed@{\pgfmathresult}%
                    \pgfmath@smuggleone\pgfmathresult
                    \endgroup
    }}}}%
}

%%%%%%%%%%%%%%%%%%%%%%%%%%%%%%
%% subcircuits (experimental)
%%
%% introduced by Romano Giannetti around April 2021
%% changes suggested by Jonathan P. Spratte
%%
\newbox\ctikz@scratchbox
\pgfutil@protected\def\ctikzsubcircuitdef#1#2#3{%
    \expandafter\gdef\csname #1@Anchor\endcsname{}%
    \expandafter\gdef\csname #1@setanchors\endcsname{%
        \setbox\ctikz@scratchbox=\hbox{%
        \tikzpicture
        \draw (0,0) \csname#1\endcsname{T-#1}{};
        \foreach [count=\i] \anchor in {#2}
        % reference anchor is -center
        \draw (0,{2-\i/2}) let \p1 = ($(T-#1-subckt@reference)-(T-#1-\anchor)$) in
            node[right]{\anchor: \x1,\y1 \expandafter\xdef\csname #1@Anchor\anchor\endcsname{++(\x1,\y1)}};
        \endtikzpicture
        }%
    }%
    \expandafter\gdef\csname#1\endcsname##1##2{%
        \csname #1@Anchor##2\endcsname coordinate(##1-subckt@reference)#3%
    }%
}
\long\def\ctikzsubcircuitactivate#1{\csname #1@setanchors\endcsname}


%%%%%%%%%%%%%%%%%%%%%%%%%%%%%%
%% Basic utility macros
%%
%% Functions provided here are:
%%  \pgf@circ@ifempty{<argument>}{<true>}{<false>}
%%    Tests whether <argument> is completely empty.
%%  \pgf@circ@ifblank{<argument>}{<true>}{<false>}
%%    Tests whether <argument> is either empty or only contains spaces.
%%  \pgf@circ@trimspaces@do{<argument>}{<next>}
%%    Trims at most one space from either end of <argument> and forwards the
%%    result to <next> as <next>{<trimmed argument>}

% these two are pretty standard code
\long\def\pgf@circ@ifempty#1%
  {%
    \if\relax\detokenize{#1}\relax
      \expandafter\pgfutil@firstoftwo
    \else
      \expandafter\pgfutil@secondoftwo
    \fi
  }
\long\def\pgf@circ@ifblank#1%
  {%
    \if\relax\detokenize\expandafter{\pgfutil@gobble#1.}\relax
      \expandafter\pgfutil@firstoftwo
    \else
      \expandafter\pgfutil@secondoftwo
    \fi
  }

% \pgfutil@trimspaces needs two expansions. The first expansion we'll do during
% the definition.
\def\pgf@circ@trimspaces@do#1%
  {%
    \def\pgf@circ@trimspaces@do##1%
      {\expandafter\pgf@circ@trimspaces@do@\expandafter{#1}}%
  }
\expandafter\pgf@circ@trimspaces@do\expandafter{\pgfutil@trimspaces{#1}}
\def\pgf@circ@trimspaces@do@#1#2{#2{#1}}

%%%%%%%%%%%%%%%%%%%%%%%%%%%%%%
%% list handling
%%
%% Contribution by Jonathan P. Spratte (blame him!)
%%
%% The list implementation here has a few limitations. Those are:
%%  1. not long, so no \par in the lists (but many used functions in pgfutil
%%     aren't long as well)
%%  2. list elements can't contain a Q with category code 3 (but the used
%%     function \pgfutil@trimspaces doesn't support them as well, and this
%%     should be a very rare token anyway)
%%  3. list elements can't contain the token \pgf@circ@set@list as that is used
%%     as the end marker
%%  4. currently these lists are meant for numeric data (hence only
%%     \pgf@circ@if@num@in@list is provided as a test), as a result there is
%%     another limitation for the data here. If the list element contains no
%%     hyphen '-', the element will be stored without further processing
%%     'as-is' (well, after trimming spaces). If there is a hyphen we assume
%%     well-behaved input data and will interpret this as a num-range without
%%     further tests.
%%
%% Functions provided here are:
%%  \pgf@circ@set@list<macro>{<csv-list>}
%%    Parses the <csv-list> and stores the result inside <macro> (local
%%    assignment). If a num-range given as <start - stop> (with or without
%%    spaces) is found it will be expanded to <start>,<start+1>,...,<stop>.
%%  \pgf@circ@if@num@in@list<macro>{<value>}{<true>}{<false>}
%%    Checks whether <value> (numeric value, evaluated once with \numexpr) is
%%    found inside the list stored in <macro>. There are two special cases: If
%%    <macro> is undefined <false> is executed. If the list contains only one
%%    element and that is `all' <true> is executed.

% set the catcode of our marker
\chardef\pgf@circ@temp=\the\catcode`\Q
\catcode`\Q=3

% lists will have the structure
% <marker><element 1><marker>...<element n><marker>
% As marker we use a Q with category 3. Under the assumption that no list
% element does ever contain that token we can set the elements without braces,
% allowing us to use \pgfutil@in@ to search for elements (see above). The other
% token that isn't allowed to show up in the list is \pgf@circ@set@list, that we
% use as another marker during parsing.
% The other big restriction in this implementation is that lists can't contain a
% \par (but \pgfutil@in@ doesn't support that anyway so there is not much
% sense in supporting it here)
\pgfutil@protected\def\pgf@circ@set@list#1#2%
  {%
    % clear the error flag
    \let\pgf@circ@error@marker\pgf@circ@undefined
    % set the list
    \edef#1%
      {Q\pgf@circ@set@list@sanitize#2,\pgf@circ@set@list,\pgf@circ@set@list}%
    % there was an error, throw the error message, recovery was already done by
    % ignoring the offending elements.
    \ifx\pgf@circ@error@marker\relax
      \begingroup
        \newlinechar`\^^J
        \pgfutil@packageerror{circuitikz}
          {%
            Unallowed marker found in list^^J%
            \pgfutil@unexpanded{#2}.^^J%
            Offending elements were ignored.%
          }
          {Lists can't contain a Q with category code 3}%
      \endgroup
    \fi
  }
% just a utility for the <marker> test
\def\pgf@circ@set@list@gobbletomarker#1Q{}
% quick way to check whether list parsing is done by gobbling up to a marker, in
% this case the marker is \pgf@circ@set@list
\def\pgf@circ@set@list@sanitize@checkend#1\pgf@circ@set@list{}
% will only be called after the last element is handled, will gobble the
% remainder of the current sanitizing step
\def\pgf@circ@set@list@sanitize@end\pgf@circ@set@list#1\pgf@circ@set@list{}
% grabs the next list element, checks whether we're done, and if not sanitizes
% it (meaning stripping spaces from either end and placing the <marker>).
\def\pgf@circ@set@list@sanitize#1,%
  {%
    \pgf@circ@set@list@sanitize@checkend
      #1\pgf@circ@set@list@sanitize@end\pgf@circ@set@list
    \pgf@circ@set@list@sanitize@a{#1}%
  }
% testing whether a list element contains the used <marker> expandably, if it
% does set the flag, else continue sanitizing.
\def\pgf@circ@set@list@sanitize@a#1%
  {%
    \expandafter\pgf@circ@ifempty\expandafter
      % if this is empty no marker was found
      {\pgf@circ@set@list@gobbletomarker#1Q}
      {%
        \pgf@circ@ifblank{#1}
          {}% ignore blank entries
          {\pgf@circ@trimspaces@do{#1}\pgf@circ@set@list@sanitize@b}%
      }
      {%
        % panic, there was a marker found in a list element. We'll recover by
        % ignoring the current element after setting a flag. When we do
        % \csname ...\endcsname on an undefined macro name TeX will let that
        % macro to relax and we exploit this to expandable set a flag and
        % gobbling the result.
        \expandafter\pgfutil@gobble\csname pgf@circ@error@marker\endcsname
      }%
    % get the next element
    \pgf@circ@set@list@sanitize
  }
% we'll protect any element from further expanding using \unexpanded and place
% the marker after the element, and ignore empty/blank elements
\def\pgf@circ@set@list@sanitize@b#1%
  {%
    \expandafter\pgf@circ@ifempty\expandafter
      % if this is empty no hyphen is found
      {\pgf@circ@gobbletohyphen#1-}
      {\pgfutil@unexpanded{#1}Q}
      {\pgf@circ@set@list@parse@range{#1}}%
  }
\def\pgf@circ@gobbletohyphen#1-{}
\def\pgf@circ@set@list@parse@range#1{\pgf@circ@set@list@parse@range@a#1Q}
\def\pgf@circ@set@list@parse@range@a#1-#2Q%
  {%
    \expandafter\pgf@circ@set@list@parse@range@b
      \the\numexpr#1\expandafter Q\the\numexpr#2Q%
  }
\def\pgf@circ@set@list@parse@range@b#1Q#2Q%
  {%
    \ifnum#1<#2
      % expand to the range from #1 to #2 (inclusive)
      \pgf@circ@set@list@range{#1}{#2}%
    \else
      \ifnum#2<#1
        % if #2 is smaller than #1 just swap the order
        \pgf@circ@set@list@range{#2}{#1}%
      \else
        % last case, they are equal, so just put the result here
        #1Q%
      \fi
    \fi
  }
\def\pgf@circ@set@list@parse@range@norange#1Q#2{\pgfutil@unexpanded{#2}Q}
\def\pgf@circ@set@list@range#1#2%
  {%
    #1Q%
    \ifnum#1<#2
      \expandafter\pgfutil@secondoftwo
    \fi
    \pgfutil@gobble
    {\expandafter\pgf@circ@set@list@range\expandafter{\the\numexpr#1+1}{#2}}%
  }

% flag for special value
\def\pgf@circ@all@flag{QallQ}
\pgfutil@protected\def\pgf@circ@if@num@in@list#1#2%
  {%
    % test whether the list macro is defined, if it isn't result is false
    \pgfutil@ifx\pgf@circ@undefined#1%
      {\pgfutil@secondoftwo}
      {%
        % test whether the list macro is just the special value 'all', if so
        % true, else search (and start that by evaluating a numexpr)
        \pgfutil@ifx\pgf@circ@all@flag#1%
          {\pgfutil@firstoftwo}
          {%
            \expandafter\pgf@circ@if@num@in@list@a\expandafter
              {\the\numexpr#2}%
              #1%
          }%
      }%
  }
% next step is expanding the list macro
\pgfutil@protected\def\pgf@circ@if@num@in@list@a#1#2%
  {\expandafter\pgf@circ@if@num@in@list@b\expandafter{#2}{#1}}
% now use \pgfutil@in@ to check whether there is the searched list element
\pgfutil@protected\def\pgf@circ@if@num@in@list@b#1#2%
  {%
    \begingroup
      % put the <marker> around the number to make sure only full matches are
      % found. \pgfutil@in@ will set \ifpgfutil@in@ to true if it finds a match
      \pgfutil@in@{Q#2Q}{#1}%
      \expandafter
    \endgroup
    \ifpgfutil@in@
      \expandafter\pgfutil@firstoftwo
    \else
      \expandafter\pgfutil@secondoftwo
    \fi
  }

% reset the catcode of Q
\catcode`\Q=\pgf@circ@temp

%%%%%%%%%%%%%%%%%%%%%%%%%%%%%%
%% temporary fix for old TikZ versions (remove me)
%%
%% All blame to Romano Giannetti for this code!
%%
%% This tries to be smart and provide \pgfutil@unexpanded and \pgfutil@ifx if
%% PGF doesn't provide them.

\ifx\pgfutil@unexpanded\pgf@circ@undefined
  \ifpgfutil@format@is@context
    \let\pgfutil@unexpanded\normalunexpanded
  \else
    \let\pgfutil@unexpanded\unexpanded
  \fi
\fi

\ifx\pgfutil@ifx\pgf@circ@undefined
  \long\def\pgfutil@ifx#1#2{%
    \ifx#1#2%
      \expandafter\pgfutil@firstoftwo
    \else
      \expandafter\pgfutil@secondoftwo
    \fi}
\fi

%%
%% generic tunable arrow for components that have no "variable" thing
%%
\def\ctikztunablearrow{\pgfutil@ifnextchar[{\ctikztunablearrow@full}{\ctikztunablearrow@simple}}%
\def\ctikztunablearrow@simple{\ctikztunablearrow@full[]}%
\def\ctikztunablearrow@full[#1]#2#3#4#5{%
    % add tunable arrow to a component
    % relative thickness, relative length, rotation from axis, name of the component
    \scope
    \draw
    \pgfextra{\pgfcirc@set@arrows{tunable}{}{latexslim}
    \pgfsetlinewidth{#2\pgflinewidth}} [#1]
        let \p1=($(#5.north east)-(#5.south west)$), \p2=($(#5.east)-(#5.west)$),
        \n1 = {veclen(\x1,\y1)},
        \n2 = {atan2(\y2,\x2)} in
        % node[above]{\n1, \n2}
        % notice that some node has the "center" on one side, so
        % midway from east to west is a safer bet for the center
        ($(#5.west)!0.5!(#5.east)$) ++({\n2+(#4)}:{-0.5*(\n1)*(#3)}) -- ++({\n2+(#4)}:{(\n1)*(#3)});
    \endscope
}
\endinput
