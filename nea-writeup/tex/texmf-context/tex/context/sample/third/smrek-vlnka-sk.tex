Hľa, všetko pokrýva sneh rúškom zimy;
ženy a~panny svieže, rudolíce
sa vyrojili hlučné na ulice;
ja hľadám jednu medzi tisícimi \dots

\dots

Celý svet miluje a~vzýva ženu.
Lež básnik nadovšetko. Dante božský
i~v~pekle Beatricu svoju hľadal,
o~Sandovej zas Musset verše skladal
a~Verlaine ospevoval ženské bozky \dots

\quotation{No, hľaďme, vy už k~bozkom veslujete!}

Nie pani, to tak prišlo samo sebou.
A~i~tak, bozk je vecou veľkolepou!
Prečo mu vyhýbať v~tom chladnom svete?
Rty vaše, pani \dots Lež, ach -- odpustite!
Nálada táto zimná rozmar tvorí.
Len škoda, že ten večer je tak skorý \dots
Lež prečo hľadíte tak rozpačite?

\quotation{Rozmýšľam práve, či by prijali ste,
keby vás pozvala aspoň k~čaju.
Veď sa už všetci z~parku rozchádzajú.
Kontúry mesta sú už veľmi hmlisté.}

Vidíte, i~tak by šiel možno s~vami
a~nečakal by ani na pozvanie.
Darmo, z~vás žien vždy príťažlivosť vanie.
Ja vzbĺknem pri ženách jak otep slamy.

\dots

Som iba básnik, človek nepraktický,
parazit sŕdc, čo šťastie loví vždycky,
ale sám záruk šťastia sotva dáva.

\dots
