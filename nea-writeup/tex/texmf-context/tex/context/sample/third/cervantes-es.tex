En un lugar de la Mancha, de cuyo nombre no quiero acordar-me, no ha
mucho tiempo que vivía un hidalgo de los de lanza en astillero, adarga
antigua, rocín flaco y galgo corredor. Una olla de algo más vaca que
carnero, salpicón las más noches, duelos y quebrantos los sábados,
lantejas los viernes, algún palomino de añadidura los domingos,
consumían las tres partes de su hacienda.
