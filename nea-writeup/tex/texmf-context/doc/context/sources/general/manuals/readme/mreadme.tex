% interface=en language=uk

% author    : Hans Hagen
% copyright : PRAGMA ADE & ConTeXt Development Team
% license   : Creative Commons Attribution ShareAlike 4.0 International
% reference : pragma-ade.nl | contextgarden.net | texlive (related) distributions
% origin    : the ConTeXt distribution
%
% comment   : Because this manual is distributed with TeX distributions it comes with a rather
%             liberal license. We try to adapt these documents to upgrades in the (sub)systems
%             that they describe. Using parts of the content otherwise can therefore conflict
%             with existing functionality and we cannot be held responsible for that. Many of
%             the manuals contain characteristic graphics and personal notes or examples that
%             make no sense when used out-of-context.
%
% comment   : Some chapters might have been published in TugBoat, the NTG Maps, the ConTeXt
%             Group journal or otherwise. Thanks to the editors for corrections. Also thanks
%             to users for testing, feedback and corrections.

\environment mcommon

% copied from cont-log: readme_logo

\startuseMPgraphic{titlepage}{width,height}
    numeric width ; width = \MPvar{width} ;
    numeric height ; height = \MPvar{height} ;
    numeric delta ; delta := width/10 ;
    numeric circle ; circle := 2.5delta ;
    color c ; c := (.2,.4,.6) ;
    path p, q, r ;
    p := unitsquare xscaled width yscaled height ;
    z1 = (delta,height-2delta) ;
    z2 = (width-delta,height-delta) ;
    z3 = (width/2-delta,2delta+circle) ;
    z4 = (x3,delta+circle/2) ;
    q := z1 { dir  -15 } .. z2 & z2 { dir -105 } .. z3 & z3 { dir  135 } .. z1 & cycle ;
    r := fullcircle xscaled circle yscaled (.85circle) rotated 15 shifted z4 ;
    pickup pencircle scaled (delta/1.5) ;
    fill p withcolor .50c ;
    fill q withcolor .75c ;
    fill r withcolor .75c ;
    draw p withcolor    c ;
    draw q withcolor    c ;
    pickup pencircle scaled (delta/2) ;
    draw r withcolor    c ;
    setbounds currentpicture to p ;
\stopuseMPgraphic

\startdocument
  [title={Read Me First},
   author={Hans Hagen},
   affiliation={PRAGMA ADE},
   location={Hasselt NL}]

\startsubject[title={Introduction}]

What licence suits best for a \TEX\ like system is a matter of taste. Personally
we dislike any licence that needs more than a few pages of dense legal code to
get the message across. A \TEX\ related system like \CONTEXT\ is a hybrid of
programs, scripts and|/|or macro code as well as documentation and sample code,
including graphics. \TEX\ related systems also have a long standing tradition of
providing support structures for users. In order to make support feasible, a
\TEX\ based system like \CONTEXT\ assumes a certain logic and structure in the
way the related files are named and organized in a tree structure. Even a small
change in one of the elements may let such a system behave differently than
manuals suggest. Swap a font, change some style defaults, leave out some pieces,
and users may end up in confusion. A licence does not give a user any guarantees!

In order to satisfy those responsible for distributing \CONTEXT, we need to
choose a licence that makes them feel comfortable. Unfortunately we don't feel
that comfortable with a licence that does not provide the guarantees that a
system will not be adapted in such ways that the advertised behaviour changes. On
the other hand, it is the responsibility of those distributing and extending the
system to make sure that this does not happen. However, users should not
automatically assume that what they get shipped is the same as the original,
which is why we stress that support (from our side) will only be given on
unaltered systems.

First of all, what is \CONTEXT ? It's just a bunch of macros, written in \TEX\
and \METAPOST, meant for typesetting documents. The macros are accompanied by
some scripts, written in \PERL\ (mainly the older scripts) \RUBY\ (also older
ones) and \LUA\ (the current fashion). The \CONTEXT\ distribution comes with a
few fonts, files that help manage resources (e.g.\ map files needed for \MKII),
as well as patterns (based on official ones, so this is a derived work).

The \CONTEXT\ distribution is packaged in a zip file organized in the \TDS\
structure.

\starttabulate[|lT|p|]
\NC \type {cont-tmf.zip} \NC the main distribution that has all relevant files \NC \NR
\NC \type {cont-tst.7z}  \NC a bunch of test files that can also serve as examples \NC \NR
\NC \type {cont-mpd.zip} \NC a \METAPOST\ to \PDF\ converter (not needed in \CONTEXT) \NC \NR
\NC \type {cont-ppc.zip} \NC a macro package for typesetting chemistry (not needed in \CONTEXT) \NC \NR
\NC \type {cont-sci.zip} \NC configuration files for using \CONTEXT\ in the \SCITE\ editor \NC \NR
\stoptabulate

There are two flavours of \CONTEXT: \MKII\ and \MKIV. The first one is frozen and
will not be extended. It runs on top of \PDFTEX\ or \XETEX. The \MKIV\ version is
actively developed and runs on top of \LUATEX\ (an engine that is developed
alongside \CONTEXT\ but that can also be used for other macro packages).

The documentation can be downloaded from our website or the Wiki. Some manuals
ship with source code. We might ship more source code but only when the source is
stable and clean and can serve as an example.

That said, what licence does apply? We need to distinguish between things that
resemble a program on the one hand and documentation on the other hand. We
(currently) use a different licence for either of them.

\stopsubject

\startsubject[title={The Code}]

The program code (i.e. anything not under the \type {/doc} subtree) is
distributed under the

\startnarrower
\goto{Creative Commons GNU GPL}[url(gpl-simple)]
\stopnarrower

For practical purposes distributers may also choose the \LATEX\ project licence,
which is considered to be a bit more \TEX\ friendly. (BSD alike licences also
apply well for \CONTEXT.)

In practice, users may forget about the legal part, if only because I haven't
even read (and understood) it completely myself, so let's stick to what Creative
Commons makes of it:

\startcolor[blue]
The GNU General Public License is a Free Software license. Like any Free Software
license, it grants to you the four following freedoms:

\startitemize
    \startitem
        The freedom to run the program for any purpose.
    \stopitem
    \startitem
        The freedom to study how the program works and adapt it to your needs.
    \stopitem
    \startitem
        The freedom to redistribute copies so you can help your neighbour.
    \stopitem
    \startitem
        The freedom to improve the program and release your improvements to the
        public, so that the whole community benefits.
    \stopitem
\stopitemize

You may exercise the freedoms specified here provided that you comply with the
express conditions of this license. The principal conditions are:

You must conspicuously and appropriately publish on each copy distributed an
appropriate copyright notice and disclaimer of warranty and keep intact all the
notices that refer to this License and to the absence of any warranty; and give
any other recipients of the Program a copy of the GNU General Public License
along with the Program. Any translation of the GNU General Public License must be
accompanied by the GNU General Public License.

If you modify your copy or copies of the program or any portion of it, or develop
a program based upon it, you may distribute the resulting work provided you do so
under the GNU General Public License. Any translation of the GNU General Public
License must be accompanied by the GNU General Public License.

If you copy or distribute the program, you must accompany it with the complete
corresponding machine-readable source code or with a written offer, valid for at
least three years, to furnish the complete corresponding machine-readable source
code.

Any of these conditions can be waived if you get permission from the copyright
holder.

Your fair use and other rights are in no way affected by the above.
\stopcolor

Many source filed mention the \quotation {\CONTEXT\ Development Team} as
copyright holder. This is because over time users submit patches and whenever
something happens to the main author, those involved in development can take
over without the need to bother about derived work. In the end the \CONTEXT\
group is the place to get more information.

\stopsubject

\startsubject[title={Recommendations}]

Here are a few recommendations in case you want to distribute, extend of embed
\CONTEXT\ in applications:

\startitemize

\startitem
    You can best leave the code base untouched. Most of \CONTEXT\ provides hooks
    and it's relatively easy to overload code. Leave the lower level system code
    untouched: changes may backfire when you update. Asking for more hooks is the
    best way to go.
\stopitem

\startitem
    Put your own code in the right subpaths, i.e.\ modules approved by the
    development team under \type {.../third}, and styles and whatever else under
    \type {.../user}. This way your code will not interfere with existing code
    and updating will give less problems. Keep in mind that \TEX\ systems have
    their own way and order in locating files, and the load order often matters.
\stopitem

\startitem
    Don't copy styles and change a few lines, but load the base one and
    built|/|patch on top of that. In the end you may benefit from improvements to
    the base style.
\stopitem

\startitem
    Be original. The whole idea behind \CONTEXT\ is that you can write your own
    styles. On the \CONTEXT\ mailing list as well as on the Wiki there are enough
    advanced users to help you make a start.
\stopitem

\startitem
    Don't hesitate to submit bugs reports and ask for extensions. It may even be
    that what you want is already present but yet undocumented.
\stopitem

\startitem
    If things don't work as expected, check to what extend your system matches
    the (more or less) standard. We provide so called minimal \CONTEXT\ trees
    that can serve as a reference. Because \CONTEXT\ evolves, make sure your
    system is up to date. The \CONTEXT\ garden provides ways to install and
    update the standard distribution.
\stopitem

\startitem
    The scripts can best be called using \type {mtxrun}. This lessens dependencies
    on the location in the tree and ensures upward compatibility. It also prevents
    clashes with similar scripts.
\stopitem

\startitem
    Some scripts depend on each other. Don't mess around with the existing
    functionality and names of the scripts and then feed them back into the
    standard distributions.
\stopitem

\stopitemize

\stopsubject

\startsubject[title={Documents}]

The documentation is provided under another Creative Commons licence:

\startnarrower
    \goto{Attribution ShareAlike}[url(bysa-simple)]
\stopnarrower

The manuals that are distributed with \CONTEXT\ comes with a rather liberal
license. However, we try to adapt these documents to upgrades in the (sub)systems
that they describe and derived work can therefore conflict with existing
functionality and we cannot be held responsible for that. Many of the manuals
contain characteristic graphics and personal notes or examples that make no sense
when used out|-|of|-|context.

The mentioned license says:

\startcolor[blue]
You are free:

\startitemize
    \startitem to copy, distribute, display, and perform the work \stopitem
    \startitem to make derivative works \stopitem
\stopitemize

{\sc Attribution:} You must attribute the work in the manner specified by the
author or licensor.

{\sc Share Alike:} If you alter, transform, or build upon this work, you may
distribute the resulting work only under a license identical to this one.

\startitemize
    \startitem
        For any reuse or distribution, you must make clear to others the license
        terms of this work.
    \stopitem
    \startitem
        Any of these conditions can be waived if you get permission from the
        copyright holder.
    \stopitem
\stopitemize

Your fair use and other rights are in no way affected by the above.
\stopcolor

If you distribute \CONTEXT\ and related software on electronic media as part of
\TEX\ distributions (either or not for money), you may also distribute the
manuals and their sources in electronic form, preferable as provided by the
maintainers of \CONTEXT.

Keep in mind that logos and cover designs are not meant to be copied. We provide
the source code for some manuals, but we don't always provide all graphics and
other resources. For instance, in some manuals we use commercial fonts and you
have to buy those yourself.

We provide the typeset manuals at our website. Those are the official ones. We
appreciate it if you do not to distribute manuals compiled on your own system as
substitutes. The manuals are a showcase for what \CONTEXT\ provides. Help us to
assure the quality.

\stopsubject

\startsubject[title={More information}]

We're not going to fill \mathematics{n}~pages with legal stuff, so if you want to
know more, you have to consult the web for the legalities mentioned. Here are a
few starting points:

\startlines
\goto{\url[gpl-simple]}[url(gpl-simple)]
\goto{\url[gpl-legal]}[url(gpl-legal)]
\stoplines

\startlines
\goto{\url[bysa-simple]}[url(bysa-simple)]
\goto{\url[bysa-legal]}[url(bysa-legal)]
\stoplines

\CONTEXT\ itself can be fetched from the main site or the garden:

\startlines
\goto{\url[pragma]}[url(pragma)]
\goto{\url[install]}[url(install)]
\stoplines

These always ship the latest versions. Alternatively you can install the whole
\TEX\ distribution, which is a yearly snapshot:

\startlines
\goto{\url[texlive]}[url(texlive)]
\stoplines

A starting point for support can be found at:

\startlines
\goto{\url[list]}[url(list)]
\goto{\url[garden]}[url(garden)]
\stoplines

And of course there is the \CONTEXT\ group:

\startlines
\goto{\url[group]}[url(group)]
\stoplines

Bugs and feature requests can be registered at the collector:

\startlines
\goto{\url[collector]}[url(collector)]
\stoplines

Releases are announced at:

\startlines
\goto{\url[announce]}[url(announce)]
\stoplines

The developers can be met at:

\startlines
\goto{\url[development]}[url(development)]
\stoplines

\stopsubject

\startsubject[title={Disclaimer}]

To play safe we include a disclaimer here, taken from the BSD style licence. For
some reason such a text is always in unreadable capitals, so \unknown

\start \txx \blue
THIS SOFTWARE IS PROVIDED BY THE AUTHOR \quotation {AS IS} AND ANY EXPRESS OR
IMPLIED WARRANTIES, INCLUDING, BUT NOT LIMITED TO, THE IMPLIED WARRANTIES
OF MERCHANTABILITY AND FITNESS FOR A PARTICULAR PURPOSE ARE DISCLAIMED.
IN NO EVENT SHALL THE AUTHOR BE LIABLE FOR ANY DIRECT, INDIRECT,
INCIDENTAL, SPECIAL, EXEMPLARY, OR CONSEQUENTIAL DAMAGES (INCLUDING, BUT
NOT LIMITED TO, PROCUREMENT OF SUBSTITUTE GOODS OR SERVICES; LOSS OF USE,
DATA, OR PROFITS; OR BUSINESS INTERRUPTION) HOWEVER CAUSED AND ON ANY
THEORY OF LIABILITY, WHETHER IN CONTRACT, STRICT LIABILITY, OR TORT
(INCLUDING NEGLIGENCE OR OTHERWISE) ARISING IN ANY WAY OUT OF THE USE OF
THIS SOFTWARE, EVEN IF ADVISED OF THE POSSIBILITY OF SUCH DAMAGE.
\stop

\unknown\ and don't bother discussing licence issues and related things with us
for the mere sake of discussing licence stuff.

\stopsubject

\stopdocument
