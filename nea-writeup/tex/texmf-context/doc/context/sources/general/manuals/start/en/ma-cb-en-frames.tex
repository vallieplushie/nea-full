\startcomponent ma-cb-en-frames

\enablemode[**en-us]

\project ma-cb

\startchapter[title=Outlined text]

\index{outline+text}

\Command{\tex{framed}}
\Command{\tex{setupframed}}
\Command{\tex{inframed}}

You can \inframed{outline} a text with \type{\framed}. The
command looks like this:

\shortsetup{framed}

The bracket pair is optional and contains the set up parameters. The curly braces
enclose the text. To be honest, the outlined text in the first paragraph was done
with \type{\inframed}. This command takes care of the interline spacing.

Some other examples of \type{\framed} and its set up parameters are shown below.

\startbuffer[a]
\framed
  [height=fit,
   width=.5\textwidth]
  {Hasselt}
\stopbuffer

\placefigure[right,none][]{}{\externalfigure[a][type=buffer]}
\typebuffer[a]

\startbuffer[b]
\framed
  [height=3em,
   width=.5\textwidth]
  {Hasselt now has more space}
\stopbuffer

\placefigure[right,none][]{}{\externalfigure[b][type=buffer]}
\typebuffer[b]

\startbuffer[d]
\framed
  [height=3em,
   width=.5\textwidth,
   foregroundcolor=red,
   framecolor=blue]
  {Hasselt now has some color}
\stopbuffer

\placefigure[right,none][]{}{\externalfigure[d][type=buffer]}
\typebuffer[d]

\startbuffer[e]
\framed
  [height=3em,
   width=.5\textwidth,
   foregroundcolor=red,
   framecolor=blue,
   rulethickness=2pt]
  {Hasselt now has more frame}
\stopbuffer

\placefigure[right,none][]{}{\externalfigure[e][type=buffer]}
\typebuffer[e]

\startbuffer[f]
\framed
  [height=3em,
   width=.5\textwidth,
   foregroundcolor=red,
   framecolor=blue,
   rulethickness=2pt,
   background=color,
   backgroundcolor=green]
  {Hasselt now has a colorful background}
\stopbuffer

\placefigure[right,none][]{}{\externalfigure[f][type=buffer]}
\typebuffer[f]

\startbuffer[g]
\framed
  [height=3em,
   width=.5\textwidth,
   foregroundcolor=red,
   framecolor=blue,
   rulethickness=2pt,
   background=color,
   backgroundcolor=green,
   foregroundstyle=bold]
  {Hasselt now has another style}
\stopbuffer

\placefigure[right,none][]{}{\externalfigure[g][type=buffer]}
\typebuffer[g]

\startbuffer[needed-for-h]
\definecolor[a][black]
\definecolor[b][white]

\startuniqueMPgraphic{LinearShade}
  fill OverlayBox
    withshademethod "linear" withcolor \MPcolor{a} shadedinto \MPcolor{b} ;
\stopuniqueMPgraphic

\defineoverlay
  [linear shade]
  [\uniqueMPgraphic{LinearShade}]
\stopbuffer

\getbuffer[needed-for-h]

\startbuffer[h]
\framed
  [height=3em,
   width=.5\textwidth,
   foregroundcolor=red,
   framecolor=blue,
   rulethickness=2pt,
   background=linear shade,
   foregroundstyle=bold]
  {Hasselt now has a little shade}
\stopbuffer

\placefigure[right,none][]{}{\externalfigure[h][type=buffer]}
\typebuffer[h]

The shady background was defined with:

\typebuffer[needed-for-h]

The \type{\framed} command is very sophisticated and is used in many macros.
The command to set up frames is:

\shortsetup{setupframed}

\stopchapter

\stopcomponent
