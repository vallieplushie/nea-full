% language=uk
%
% copyright=pragma-ade readme=readme.pdf licence=cc-by-nc-sa

\startcomponent metafun-reference

\environment metafun-environment

\startchapter[reference=reference,title={Reference}]

\startintro

In this chapter, we will demonstrate most of the drawing related primitives and
macros as present in plain \METAPOST\ and \METAFUN\ extensions.

If a path is shown and|/|or a transformation is applied, we show the original in
red and the transformed path or point in yellow. The small dark gray crosshair is
the origin and the black rectangle the bounding box. In some drawings, in light
gray we show the points that make up the path.

This list describes traditional \METAPOST\ and the stable part of \METAFUN. As
\METAPOST\ evolves, new primitives are added but they are not always that
relevant to us. If you browse the \METAFUN\ sources you will for sure notice more
functionality than summarized here. Most of that is meant for usage in \CONTEXT\
and not exposed to the user. Other macros are still somewhat experimental but
might become more official at some point. The same is true for \METAFUN\
commands: not all make sense for daily usage and some are just there as helper
for additional modules.

\stopintro

\startsection[title={Paths}]

\index{paths}

\ShowSampleA {mc} {pair}                  {(1,.5)}
\ShowSampleA {mm} {pair .. pair}          {(0,0)..(.75,0)..(1,.25)..(1,1)}
\ShowSampleA {mm} {pair ... pair}         {(0,0)..(.75,0)...(1,.25)..(1,1)}
\ShowSampleA {mm} {pair -- pair\quad (a)} {(0,0)--(.75,0)--(1,.25)--(1,1)}
\ShowSampleA {mm} {pair -- pair\quad (b)} {(0,0)..(.75,0)--(1,.25)..(1,1)}
\ShowSampleA {mm} {pair --- pair}         {(0,0)..(.75,0)---(1,.25)..(1,1)}

\ShowSampleA {mm} {pair softjoin pair}      {(0,0)..(.75,0) softjoin (1,.25)..(1,1)}
\ShowSampleA {mp} {controls pair}           {(0,0)..controls (.75,0)..(1,1)}
\ShowSampleA {mp} {controls pair and pair}  {(0,0)..controls (.75,0) and (1,.25)..(1,1)}
\ShowSampleA {mp} {tension numeric}         {(0,0)..(.75,0)..tension 2.5..(1,.25)..(1,1)}
\ShowSampleA {mp} {tension num.. and num..} {(0,0)..(.75,0)..tension 2.5 and 1.5..(1,.25)..(1,1)}
\ShowSampleA {mp} {tension atleast numeric} {(0,0)..(.75,0)..tension atleast 1..(1,.25)..(1,1)}

\ShowSampleA {mp} {cycle}        {(0,0)--(1,0)--(1,1)--cycle}
\ShowSampleA {mp} {curl numeric} {(0,0){curl 1}..(.75,0)..(1,.25)..(1,1)}
\ShowSampleA {mp} {dir numeric}  {(0,0){dir 30}..(1,1)}
\ShowSampleA {mm} {left}         {(0,0){left}..(1,1)}
\ShowSampleA {mm} {right}        {(0,0){right}..(1,1)}
\ShowSampleA {mm} {up}           {(0,0){up}..(1,1)}
\ShowSampleA {mm} {down}         {(0,0){down}..(1,1)}

\ShowSampleA {mp} {path \& path} {(0,0)..(.75,.25) \& (.75,.25)..(1,1)}

\ShowSampleA {mv} {unitvector}  {origin--unitvector(1,1)}
\ShowSampleA {mp} {dir}         {origin--dir(45)}
\ShowSampleA {mp} {angle}       {origin--dir(angle(1,1))}

\ShowSampleA {mv} {fullcircle}  {fullcircle}
\ShowSampleA {fv} {unitcircle}  {unitcircle}
\ShowSampleA {fv} {fullsquare}  {fullsquare}
\ShowSampleA {mv} {unitsquare}  {unitsquare}
\ShowSampleA {fv} {fulltriangle}{fulltriangle}
\ShowSampleA {fv} {unittriangle}{unittriangle}
\ShowSampleA {fv} {fulldiamond} {fulldiamond}
\ShowSampleA {fv} {unitdiamond} {unitdiamond}

\ShowSampleA {mv} {halfcircle}    {halfcircle}
\ShowSampleA {mv} {quartercircle} {quartercircle}

\ShowSampleA {fv} {llcircle} {llcircle}
\ShowSampleA {fv} {lrcircle} {lrcircle}
\ShowSampleA {fv} {urcircle} {urcircle}
\ShowSampleA {fv} {ulcircle} {ulcircle}
\ShowSampleA {fv} {tcircle}  {tcircle}
\ShowSampleA {fv} {bcircle}  {bcircle}
\ShowSampleA {fv} {lcircle}  {lcircle}
\ShowSampleA {fv} {rcircle}  {rcircle}

\ShowSampleA {fv} {triangle}      {triangle}
\ShowSampleA {fv} {righttriangle} {righttriangle}
\ShowSampleA {fv} {uptriangle}    {uptriangle}
\ShowSampleA {fv} {lefttriangle}  {lefttriangle}
\ShowSampleA {fv} {downtriangle}  {downtriangle}

\ShowSampleA {fv} {lltriangle} {lltriangle}
\ShowSampleA {fv} {lrtriangle} {lrtriangle}
\ShowSampleA {fv} {urtriangle} {urtriangle}
\ShowSampleA {fv} {ultriangle} {ultriangle}

\ShowSampleA {mm} {flex(pair,pair,pair)}
                  {flex ((0,0),(1,1),(1,0))}
\ShowSampleA {mm} {superellipse(pair,p..,p..,p..,num..)}
                  {superellipse((1,.5),(.5,1),(0,.5),(.5,0),.75)}

\ShowSampleA {fm} {path smoothed numeric/pair}           {unitsquare scaled 1.5 smoothed .2}
\ShowSampleA {fm} {path cornered numeric/pair}           {lltriangle scaled 1.5 cornered .2}
\ShowSampleA {fm} {path superellipsed numeric}           {unitsquare scaled 1.5 superellipsed .75}
\ShowSampleA {fm} {path randomized numeric/pair}         {unitsquare scaled 1.5 randomized (.2,.2)}
\ShowSampleA {fm} {path randomizedcontrols numeric/pair} {fullcircle scaled 1.5 randomizedcontrols (.2,.2)}
\ShowSampleA {fm} {path squeezed numeric/pair}           {unitsquare scaled 1.5 squeezed (.2,.1)}
\ShowSampleA {fm} {path snapped numeric/pair}            {fullcircle scaled 1.5 snapped (.2,.1)}

\ShowSampleB {fm} {punked path}
         {unitcircle scaled 1.5}
  {punked unitcircle scaled 1.5}

\ShowSampleB {fm} {curved path}
         {((0,0)--(.2,1)--(1,.2)--cycle)}
  {curved ((0,0)--(.2,1)--(1,.2)--cycle)}

\ShowSampleB {fm} {laddered path}
           {((0,0)--(1.4,.8)--(2.8,1.2)--(6.2,1.6))}
  {laddered ((0,0)--(1.4,.8)--(2.8,1.2)--(6.2,1.6))}

\ShowSampleB {fm} {path paralleled distance}
           {((0,0)--(5,1))}
           {((0,0)--(5,1)) paralleled .25}

\ShowSampleB {fm} {shortened path}
    {((0,0)--(6,1))}
    {((0,0)--(6,1)) shortened 1}

\ShowSampleB {fm} {unspiked path}
           {((0,0)--(1,0)--(1,1)--(2,1)--(1,1)--(0,1)) shifted (-3,0)}
  {unspiked ((0,0)--(1,0)--(1,1)--(2,1)--(1,1)--(0,1))}

\ShowSampleB {fm} {simplified path}
             {((0,0)--(1,0)--(2,0)--(2,1)--(0,1)--cycle) shifted (-3,0)}
  {simplified ((0,0)--(1,0)--(2,0)--(2,1)--(0,1)--cycle)}

\ShowSampleB {fm} {path blownup numeric/pair}
             {fullcircle scaled 1.5}
            {(fullcircle scaled 1.5) blownup .1}

\ShowSampleB {fm} {path stretched numeric/pair\quad (a)}
             {fullcircle scaled 1.5}
            {(fullcircle scaled 1.5) stretched (1.1,0.8)}

\ShowSampleB {fm} {path stretched numeric\quad (b)}
             {((0,0)--(1,1))}
             {((0,0)--(1,1)) stretched 1.5}

\ShowSampleB {fm} {path xstretched numeric}
             {fullcircle}
             {fullcircle xstretched 5}

\ShowSampleB {fm} {path ystretched numeric}
             {fullcircle}
             {fullcircle ystretched 1.5}

\ShowSampleB {fm} {path enlonged numeric}
             {((0,0)--(1,1))}
             {((0,0)--(1,1)) enlonged 1.5}

\ShowSampleB {fm} {path shorted numeric}
             {((0,0)--(2,2))}
             {((0,0)--(2,2)) shortened 0.5}

\ShowSampleA {fm} {roundedsquare(num..,num..,num..)}
                  {roundedsquare(2,1,.2)}

\ShowSampleA {fm} {tensecircle(num..,num..,num..)}
                  {tensecircle(2,1,.2)}

\ShowSampleA {fm} {pair crossed size}
                  {origin crossed 1}

\ShowSampleA {fm} {path crossed size}
                  {fullcircle scaled 2 crossed .5}

\ShowSampleA {fm} {(constructed)function}
                  {constructedfunction("--")("x","sin(x)",0,2pi,pi/10)}

\ShowSampleA {fm} {curvedfunction}
                  {curvedfunction("x","sin(x)",0,2pi,pi/10)}

\ShowSampleA {fm} {straightfunction}
                  {straightfunction("x","sin(x)",0,2pi,pi/10)}

\ShowSampleA {fm} {constructedpath}
                  {constructedpath("..")((0,0),(1,2),(2,1),(3,2))}

\ShowSampleA {fm} {curvedpath}
                  {curvedpath((0,0),(1,2),(2,1),(3,2))}

\ShowSampleA {fm} {straightpath}
                  {straightpath((0,0),(1,2),(2,1),(3,2))}

\ShowSampleA {fm} {leftarrow}
                  {leftarrow(fullcircle,3,2)}

\ShowSampleA {fm} {rightarrow}
                  {rightarrow(fullcircle,3,2)}

\ShowSampleA {fm} {centerarrow}
                  {centerarrow(fullcircle,3,2)}

\ShowSampleX {fm} {arrowhead} {draw arrowhead fullcircle}
\ShowSampleX {fm} {arrowpath} {draw arrowpath fullcircle}

\ShowSampleA {mm} {buildcycle}
                  {buildcycle(fullcircle,fullsquare)}

\ShowSampleA {fm} {circularpath} {circularpath(4)}
\ShowSampleA {fm} {squarepath}   {squarepath(4)}
\ShowSampleA {fm} {linearpath}   {linearpath(4)}

\stopsection

\startsection[title={Transformations}]

\index{transformations}

\ShowSampleB {mp} {path scaled numeric}        {fullcircle} {fullcircle scaled .50}
\ShowSampleB {mp} {path xscaled numeric}       {fullcircle} {fullcircle xscaled .25}
\ShowSampleB {mp} {path yscaled numeric}       {fullcircle} {fullcircle yscaled .25}
\ShowSampleB {mp} {path zscaled pair}          {fullcircle} {fullcircle zscaled (2,.25)}
\ShowSampleB {mp} {path xyscaled numeric/pair} {fullcircle} {fullcircle xyscaled (.5,.7)}
\ShowSampleB {mp} {path xyscaled pair}         {fullcircle} {fullcircle xyscaled (2,.25)}
\ShowSampleB {mp} {path shifted pair}          {fullcircle} {fullcircle shifted (2,.25)}

\ShowSampleB {fm} {path leftenlarged   numeric} {fullsquare} {fullsquare leftenlarged .25}
\ShowSampleB {fm} {path topenlarged    numeric} {fullsquare} {fullsquare topenlarged .25}
\ShowSampleB {fm} {path rightenlarged  numeric} {fullsquare} {fullsquare rightenlarged .25}
\ShowSampleB {fm} {path bottomenlarged numeric} {fullsquare} {fullsquare bottomenlarged .25}

\ShowSampleB {fm} {path enlarged numeric}   {fullcircle} {fullcircle enlarged .25}
\ShowSampleB {fm} {path enlarged pair}      {fullcircle} {fullcircle enlarged (1,.25)}
\ShowSampleB {fm} {path llenlarged numeric} {fullcircle} {fullcircle llenlarged .25}
\ShowSampleB {fm} {path lrenlarged numeric} {fullcircle} {fullcircle lrenlarged .25}
\ShowSampleB {fm} {path urenlarged numeric} {fullcircle} {fullcircle urenlarged .25}
\ShowSampleB {fm} {path ulenlarged numeric} {fullcircle} {fullcircle ulenlarged .25}
\ShowSampleB {fm} {path llenlarged pair}    {fullcircle} {fullcircle llenlarged (1,.25)}
\ShowSampleB {fm} {path lrenlarged pair}    {fullcircle} {fullcircle lrenlarged (1,.25)}
\ShowSampleB {fm} {path urenlarged pair}    {fullcircle} {fullcircle urenlarged (1,.25)}
\ShowSampleB {fm} {path ulenlarged pair}    {fullcircle} {fullcircle ulenlarged (1,.25)}
\ShowSampleB {fm} {path llmoved numeric}    {fullcircle} {fullcircle llmoved .25}
\ShowSampleB {fm} {path lrmoved numeric}    {fullcircle} {fullcircle lrmoved .25}
\ShowSampleB {fm} {path urmoved numeric}    {fullcircle} {fullcircle urmoved .25}
\ShowSampleB {fm} {path ulmoved numeric}    {fullcircle} {fullcircle ulmoved .25}
\ShowSampleB {fm} {path llmoved pair}       {fullcircle} {fullcircle llmoved (1,.25)}
\ShowSampleB {fm} {path lrmoved pair}       {fullcircle} {fullcircle lrmoved (1,.25)}
\ShowSampleB {fm} {path urmoved pair}       {fullcircle} {fullcircle urmoved (1,.25)}
\ShowSampleB {fm} {path ulmoved pair}       {fullcircle} {fullcircle ulmoved (1,.25)}
\ShowSampleB {mp} {path slanted numeric}    {fullcircle} {fullcircle slanted .5}
\ShowSampleB {mp} {path rotated numeric}    {fullsquare} {fullsquare rotated 45}

\ShowSampleB {mm} {path rotatedaround(pair,numeric)}  {fullsquare} {fullsquare rotatedaround((.25,.5),45)}
\ShowSampleB {mm} {path reflectedabout(pair,pair)}    {fullcircle} {fullcircle reflectedabout((.25,-1),(.25,+1))}
\ShowSampleB {mp} {reverse path}                      {fullcircle} {reverse fullcircle shifted(.5,0)}
\ShowSampleB {mm} {counterclockwise path}             {fullcircle} {counterclockwise fullcircle shifted(.5,0)}
\ShowSampleB {mm} {tensepath path}                    {fullcircle} {tensepath fullcircle}

\ShowSampleB {mp} {subpath (numeric,numeric) of path} {fullcircle} {subpath (1,5) of fullcircle}
\ShowSampleB {mm} {path cutbefore pair} {fullcircle} {fullcircle cutbefore point 3 of fullcircle}
\ShowSampleB {mm} {path cutafter  pair} {fullcircle} {fullcircle cutafter point 3 of fullcircle}
\ShowSampleB {mm} {path cutends .1} {fullcircle} {fullcircle cutends .5}

\ShowSampleC {mp} {llcorner path} {fullcircle} {llcorner fullcircle}
\ShowSampleC {mp} {lrcorner path} {fullcircle} {lrcorner fullcircle}
\ShowSampleC {mp} {urcorner path} {fullcircle} {urcorner fullcircle}
\ShowSampleC {mp} {ulcorner path} {fullcircle} {ulcorner fullcircle}
\ShowSampleC {mm} {center   path} {fullcircle} {center   fullcircle}

\ShowSampleD {fm} {boundingbox      path} {fullcircle} {boundingbox fullcircle}
\ShowSampleD {fm} {boundingcircle   path} {fullsquare} {boundingcircle fullsquare}
\ShowSampleD {fm} {innerboundingbox path} {fullcircle} {innerboundingbox fullcircle}
\ShowSampleD {fm} {outerboundingbox path} {fullcircle} {outerboundingbox fullcircle}

\ShowSampleDD {fm} {bottomboundary path} {fullcircle} {bottomboundary fullcircle}
\ShowSampleDD {fm} {leftboundary   path} {fullcircle} {leftboundary fullcircle}
\ShowSampleDD {fm} {topboundary    path} {fullcircle} {topboundary fullcircle}
\ShowSampleDD {fm} {rightboundary  path} {fullcircle} {rightboundary fullcircle}

\ShowSampleP {fm} {bbwidth path} {draw textext(decimal bbwidth (fullcircle xscaled 100 yscaled 200))}
\ShowSampleP {fm} {bbwidth path} {draw textext(decimal bbheight (fullcircle xscaled 100 yscaled 200))}

\ShowSampleE {fm} {path/picture xsized numeric}  {xsized 3cm} {currentpicture xsized 5cm}
\ShowSampleE {fm} {path/picture ysized numeric}  {ysized 2cm} {currentpicture ysized 2cm}
\ShowSampleE {fm} {path/picture xysized numeric} {xysized (3cm,2cm)} {currentpicture xysized (3cm,2cm)}

\ShowSampleP {fm}
  {area path}
  {draw area ((0,10)--(20,20)--(30,5)--(40,10)--(50,5)--(60,5))}

\ShowSampleT {mp}
  {setbounds picture}
  {draw fullcircle ; setbounds currentpicture to unitsquare}
  {draw fullcircle scaled .5 InGray; setbounds currentpicture to unitsquare scaled .5}

\ShowSampleT {mm}
  {clip path}
  {fill fullcircle ; clip currentpicture to fullsquare scaled 0.9}
  {fill fullcircle scaled 1 InRed ; clip currentpicture to fullsquare scaled 0.9}

\ShowSampleT {mm}
  {path peepholed  path}
  {fill (fullcircle peepholed fullsquare)}
  {fill (fullcircle peepholed fullsquare) InRed}

\ShowSampleT {fm}
  {anchored}
  {draw anchored.urt(textext("ll"),origin)}
  {draw anchored.urt(textext("ll") xsized (5mm/Scale),origin) InRed ;}

% \ShowSampleT {fm}
%   {autoalign}
%   {draw textext.autoalign(260)("260")}
%   {draw textext.autoalign(260)("260")}

% draw textext.autoalign(260)("\strut oeps 3") ;

\ShowSampleX {fm}
    {path crossingunder path}
    {draw (fullsquare rotated 45) crossingunder fullsquare}

\stopsection

\startsection[title={Points}]

\index{points}

%ShowSampleF {mp} {center path}                 {fullcircle} {center fullcircle}
\ShowSampleF {mm} {top pair}                    {fullcircle} {top center fullcircle}
\ShowSampleF {mm} {bot pair}                    {fullcircle} {bot center fullcircle}
\ShowSampleF {mm} {lft pair}                    {fullcircle} {lft center fullcircle}
\ShowSampleF {mm} {rt pair}                     {fullcircle} {rt  center fullcircle}
\ShowSampleF {mp} {point       numeric of path} {fullcircle} {point 2 of fullcircle}
\ShowSampleF {fm} {point       numeric on path} {fullcircle} {point .5 on fullcircle}
\ShowSampleF {fm} {point    numeric along path} {fullcircle} {point 1cm along fullcircle}
\ShowSampleF {mp} {precontrol  numeric of path} {fullcircle} {precontrol 2 of fullcircle}
\ShowSampleF {mp} {postcontrol numeric of path} {fullcircle} {postcontrol 2 of fullcircle}
\ShowSampleF {mp} {directionpoint pair of path} {fullcircle} {directionpoint (2,3) of fullcircle}

\ShowSampleG {mc} {numeric[pair,pair]} {(1,1)} {.5[(0,0),(1,1)]}

\ShowSampleH {mm} {path intersectionpoint path} {fullcircle} {fulldiamond}
                  {fullcircle intersectionpoint fulldiamond}

\ShowSampleHH {mm} {interpath(numeric,path,path}
                   {interpath(.8,fullcircle,fullsquare)}

\ShowSampleHH {fm} {interpolated(numeric,path,path}
                   {interpolated(.8,fullcircle,fullsquare)}

\ShowSampleO {mm} {right} {draw left}
\ShowSampleO {mm} {up}    {draw up}
\ShowSampleO {mm} {left}  {draw left}
\ShowSampleO {mm} {down}  {draw down}

\stopsection

\startsection[title={Colors}]

\index{colors}

\ShowSampleI {mp} {withcolor rgbcolor} {withcolor (.625,0,0)}
\ShowSampleI {mp} {withrgbcolor rgbcolor} {withrgbcolor (.625,0,0)}
\ShowSampleI {mp} {withcmykcolor cmykcolor} {withcmykcolor (.375,0,0,0)}
\ShowSampleI {mp} {withgray / withgrey numeric} {withgray .625}
\ShowSampleI {mp} {withcolor namedcolor} {withcolor namedcolor("darkblue")}
\ShowSampleI {mp} {withcolor spotcolor} {withcolor spotcolor("tempone",red/2)}
\ShowSampleI {mp} {withcolor multitonecolor} {withcolor .2 * multitonecolor("temptwo",blue/2,yellow/3)}

Remark: at the time of this writing only Acrobat shows spot- and multitonecolors
properly. Possible indications of a viewing problem are factors not being applied
(in the page stream) or colors that are way off.

\ShowSampleU {mp} {red}     {fill fullcircle scaled 2 withcolor red/2}
\ShowSampleU {mp} {green}   {fill fullcircle scaled 2 withcolor green/2}
\ShowSampleU {mp} {blue}    {fill fullcircle scaled 2 withcolor blue/2}
\ShowSampleU {mp} {cyan}    {fill fullcircle scaled 2 withcolor cyan/2}
\ShowSampleU {mp} {magenta} {fill fullcircle scaled 2 withcolor magenta/2}
\ShowSampleU {mp} {yellow}  {fill fullcircle scaled 2 withcolor yellow/2}
\ShowSampleU {mp} {black}   {fill fullcircle scaled 2 withcolor black/2}
\ShowSampleU {mp} {white}   {fill fullcircle scaled 2 withcolor white/2}

\ShowSampleU {mp} {blackcolor} {fill fullcircle withcolor blackcolor red}

%ShowSampleI {fm} {withcolor cmyk(c,m,y,k)}  {withcolor cmyk(0,.625,.625,0)}
%ShowSampleI {fm} {withcolor transparent(n.m,color)}  {withcolor transparent(1,.625,red)}

\ShowSampleI {fm} {withtransparency(num,num)} {withcolor red withtransparency (1,.625)}

% \ShowSampleZ {fm} {withshade numeric} {Shades need to be declared before they can be (re)used.}

\startMPinclusions
    defineshade cshade withshademethod "circular" ;
    defineshade lshade withshademethod "linear" ;
\stopMPinclusions

\ShowSampleW {fm} {shaded someshade }
    {fill fullsquare shaded lshade}
    {fill fullsquare scaled 2cm shaded lshade}

This assumes the definition:

\starttyping
defineshade lshade withshademethod "linear" ;
\stoptyping

\ShowSampleW {fm} {shaded someshade}
    {fill fullcircle shaded cshade}
    {fill fullcircle scaled 2cm shaded cshade}

This assumes the definition:

\starttyping
defineshade cshade withshademethod "circular" ;
\stoptyping

%     withshadefactor 1
%     withshadedomain (0,1)
%     withshadecolors (black,white)
%     withtransparency (1,.5)

% \startMPcode
%     fill fullcircle scaled 3cm
%         shaded myshade
%         withshadefactor 0.7
%     ;
% \stopMPcode

% \startMPcode
%     fill fullcircle scaled 3cm
%         shaded myshade
%         withshadecolors (red,green)
%         withshadefactor 1
%         withtransparency (1,.75)
%     ;
% \stopMPcode

% \startMPcode
%     fill fullcircle scaled 3cm
%         shaded myshade ;
%         withshadefactor 1
%         withshadedomain (0,1)
%         withshadecolors (green,blue)
%         withtransparency (1,.5)
%     ;
% \stopMPcode

% \startMPcode
%     fill fullcircle scaled 3cm
%         shaded myshade ;
%         withshadefactor 1
%         withshadedomain (0,1)
%         withcolor blue shadedinto yellow
%         withtransparency (1,.5)
%     ;
% \stopMPcode

\ShowSampleV {mp}
    {basiccolors}
    {for i=0 upto 21 : fill ... withcolor basiccolors[i] ; endfor}
    {for i=0 upto 21 : fill fullcircle shifted (i,0) withcolor basiccolors[i] ; endfor}

\stopsection

\startsection[title={Attributes}]

\index{attributes}

\ShowSampleII {mp} {dashed withdots} {dashed withdots}
\ShowSampleII {mp} {dashed evenly}   {dashed evenly}
\ShowSampleII {mp} {dashed oddly}    {dashed oddly}
\ShowSampleII {mp} {dashpattern}     {dashed dashpattern (on .1 off .2 on .3 off .4)}
\ShowSampleII {mp} {undashed}        {dashed evenly undashed}

\ShowSampleJ {mm} {pencircle transform} {pencircle}
\ShowSampleJ {mm} {pensquare transform} {pensquare}
\ShowSampleJ {mm} {penrazor transform}  {penrazor}
\ShowSampleK {mm} {penspeck transform}  {penspeck}

\ShowSampleL {mm} {draw}         {fullcircle}
\ShowSampleL {mm} {fill}         {fullcircle}
\ShowSampleL {mm} {filldraw}     {fullcircle}
\ShowSampleL {mm} {drawfill}     {fullcircle}
\ShowSampleL {mm} {drawdot}      {origin}
\ShowSampleL {mm} {drawarrow}    {fullcircle}
\ShowSampleL {mm} {drawdblarrow} {fullcircle}

\ShowSampleL {fm} {eofill}       {fullcircle}

\ShowSampleM {mm} {undraw}     {fullcircle}
\ShowSampleM {mm} {unfill}     {fullcircle}
\ShowSampleM {mm} {unfilldraw} {fullcircle}
\ShowSampleM {mm} {undrawfill} {fullcircle}
\ShowSampleM {mm} {undrawdot}  {origin}

\ShowSampleQ {mm} {cutdraw} {origin--(1,1)}

\ShowSampleN {mv} {butt}    {linecap := butt}    {(0,.5)--(.5,0)--(1,1)}
\ShowSampleN {mv} {rounded} {linecap := rounded} {(0,.5)--(.5,0)--(1,1)}
\ShowSampleN {mv} {squared} {linecap := squared} {(0,.5)--(.5,0)--(1,1)}

\ShowSampleN {mv} {mitered} {linejoin := mitered} {(0,.5)--(.5,0)--(1,1)}
\ShowSampleN {mv} {rounded} {linejoin := rounded} {(0,.5)--(.5,0)--(1,1)}
\ShowSampleN {mv} {beveled} {linejoin := beveled} {(0,.5)--(.5,0)--(1,1)}

\ShowSampleR {fm} {inverted picture}         {inverted currentpicture}
\ShowSampleR {fm} {picture uncolored color}  {currentpicture uncolored green}
\ShowSampleR {fm} {picture softened numeric} {currentpicture softened .8}
\ShowSampleR {fm} {picture softened color}   {currentpicture softened (.7,.8,.9)}
\ShowSampleR {fm} {grayed picture}           {grayed currentpicture}

\stopsection

\startsection[title={Text}]

\index{text}

\ShowSampleO {mm} {label}           {label("MetaFun",origin)}
\ShowSampleO {mm} {label.top}   {label.top("MetaFun",origin)}
\ShowSampleO {mm} {label.bot}   {label.bot("MetaFun",origin)}
\ShowSampleO {mm} {label.lft}   {label.lft("MetaFun",origin)}
\ShowSampleO {mm} {label.rt}     {label.rt("MetaFun",origin)}
\ShowSampleO {mm} {label.llft} {label.llft("MetaFun",origin)}
\ShowSampleO {mm} {label.lrt}   {label.lrt("MetaFun",origin)}
\ShowSampleO {mm} {label.urt}   {label.urt("MetaFun",origin)}
\ShowSampleO {mm} {label.ulft} {label.ulft("MetaFun",origin)}

\ShowSampleW {mp} {dotlabel}
  {dotlabel.bot("\tttf metafun",(2cm,1cm))}
  {dotlabel.bot("\tttf metafun",(2cm,1cm))}

\ShowSampleW {mp} {dotlabels + range .. thru ..}
  {z1 = ... ; dotlabels.bot(range 1 thru 3)}
  {z1=(0,0); z2=(10mm,10mm); z3=(20mm,15mm); ; z4=(30mm,5mm); dotlabels.bot(range 1 thru 4)}

\ShowSampleW {mp} {labels + range .. thru ..}
  {z1 = ... ; labels.bot(range 1 thru 3)}
  {z1=(0,0); z2=(10mm,10mm); z3=(20mm,15mm); ; z4=(30mm,5mm); labels.bot(range 1 thru 4)}

\ShowSampleQQ {fm}
  {thelabel(string,pair)}
  {draw thelabel("MetaFun",(2cm,0))}

\ShowSampleQQ {fm}
  {formatted(string)}
  {draw textext(formatted("@0.5f",1.234))}

\ShowSampleQQ {fm}
  {format(string) : graph package}
  {draw textext(format("@5E-2",1.234))}

\ShowSampleP {mp}
  {btex text etex}
  {draw btex MetaTeX etex}

\ShowSampleQQ {fm}
  {textext(string)}
  {draw textext("MetaFun")}

\ShowSampleQQ {fm}
  {thetextext(string,pair)}
  {draw thetextext("MetaFun",(2cm,0))}

% \ShowSampleQQ {fm}
%   {graphictext string ...}
%   {graphictext "MetaFun"}

\ShowSampleQQQ {fm}
  {outlinetext.d("string")(d)}
  {draw outlinetext.d("MetaFun")(InRed)}

\ShowSampleQQQ {fm}
  {outlinetext.f("string")(f)}
  {draw outlinetext.f("MetaFun")(InYellow)}

\ShowSampleQQQ {fm}
  {outlinetext.b("string")(f)(d)}
  {draw outlinetext.b("MetaFun")(InYellow)(InRed)}

\ShowSampleQQQ {fm}
  {outlinetext.r("string")(d)(f)}
  {draw outlinetext.r("MetaFun")(InRed)(InYellow)}

\ShowSampleUU {fm}
  {outlinetext.p("string")}
  {draw outlinetext.p("MetaFun")}

\stopsection

\startsection[title={Control}]

\index{loops}

\ShowSampleU {mp} {for (positive step) until} {for i=0 step 2 until 8: drawdot (i,0) ; endfor}
\ShowSampleU {mp} {for (negative step) until} {for i=6 step -2 until 0: drawdot (i,0) ; endfor}
\ShowSampleU {mm} {for upto} {for i=0 upto 12: drawdot (i,0) ; endfor}
\ShowSampleU {mm} {for downto} {for i=10 downto 0: drawdot (i,0) ; endfor}
\ShowSampleU {mm} {forsuffixes} {forsuffixes i=1,4,6,12: drawdot (i,0) ; endfor}

\stopsection

\startsection[title={Graphics}]

\index{graphics}

\ShowSampleS {fm}
  {loadfigure string number numeric ...}
  {loadfigure "mycow.mp" number 1 scaled .25}

\ShowSampleS {fm}
  {externalfigure string ...}
  {draw externalfigure "mycow.pdf" scaled 3cm}

\ShowSampleT {fm} {addbackground text}
  {addbackground withcolor .625 yellow}
  {fill fullcircle xyscaled (2,1) InRed; addbackground InYellow}

\ShowSampleQQ {mm} {image (text)}
  {draw image(draw fullcircle) xscaled 4cm yscaled 1cm}

\ShowSampleT {mm} {decorated (text) text}
  {draw decorated (....) withcolor red}
  {draw decorated (fill fullcircle) InRed}

\ShowSampleT {mm} {undecorated (text) text}
  {draw undecorated (.... withcolor yellow) withcolor red}
  {draw undecorated (fill fullcircle InYellow) InRed}

\ShowSampleT {mm} {redecorated (text) text}
  {draw redecorated (.... withcolor yellow) withcolor red}
  {draw redecorated (fill fullcircle InYellow) InRed}

\ShowSampleS {mm}
  {bitmapimage (w,h,data)}
  {draw bitmapimage (2,2,"114477AA") rotated 15 scaled 4cm}

\ShowSampleT {mm}
  {withmask string}
  {draw externalfigure "m-1.png" scaled 2cm withmask "m-2.png"}
  {Scale := 1 ;
   draw externalfigure "m-2.png" scaled 2cm shifted (-3cm,0) ;
   draw externalfigure "m-1.png" scaled 2cm shifted (-6cm,0) ;
   draw externalfigure "m-1.png" scaled 2cm withmask "m-2.png"}

\stopsection

\stopchapter

\stopcomponent

%% draw leftpath  fullcircle scaled 1cm withpen pencircle scaled 1mm withcolor .625red ;
%% draw reverse leftpath (reverse fullcircle scaled 2cm) withpen pencircle scaled 1mm withcolor .625yellow ;
%% draw rightpath fullcircle scaled 3cm withpen pencircle scaled 1mm withcolor .625white ;
%%
%% path p ; p := (0,0) .. (1,2) .. cycle ;
%% draw leftpath  p scaled 1cm withpen pencircle scaled 1mm withcolor .625red ;
%% draw reverse leftpath (reverse p  scaled 2cm) withpen pencircle scaled 1mm withcolor .625yellow ;
%% draw rightpath p  scaled 3cm  withpen pencircle scaled 1mm withcolor .625white ;
