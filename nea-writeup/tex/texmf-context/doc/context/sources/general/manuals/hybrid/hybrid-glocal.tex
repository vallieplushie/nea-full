% language=uk

\startcomponent hybrid-glocal

\environment hybrid-environment

\startchapter[title={Glocal assignments}]

Here is a nice puzzle. Say that you do this:

\starttyping
\def\test{local} \test
\stoptyping

What will get typeset? Right, you'll get \type {local}. Now take this:

\startbuffer
\bgroup
  \def \test {local}[\test]
  \xdef\test{global}[\test]
  \def \test {local}[\test]
\egroup
                    [\test]
\stopbuffer

\typebuffer

Will you get:

\starttyping
[local] [local] [local] [global]
\stoptyping

or will it be:

\starttyping
[local] [global] [local] [global]
\stoptyping

Without knowing \TEX, there are good reasons for getting either of them: is a
global assignment global only i.e.\ does it reach over the group(s) or is it
global and local at the same time? The answer is that the global definitions also
happens to be a local one, so the second line is what we get.

Something similar happens with registers, like counters:

\startbuffer
\newcount\democount
\bgroup
         \democount 1[\the\democount]
  \global\democount 2[\the\democount]
         \democount 1[\the\democount]
\egroup
                     [\the\democount]
\stopbuffer

\typebuffer

We get: {\tttf\getbuffer\removeunwantedspaces}, so this is
consistent with macros. But how about boxes?

\startbuffer
\bgroup
         \setbox0\hbox {local}[\copy0:\the\wd0]
  \global\setbox0\hbox{global}[\copy0:\the\wd0]
         \setbox0\hbox {local}[\copy0:\the\wd0]
\egroup
                              [\copy0:\the\wd0]
\stopbuffer

\typebuffer

This gives:

\startlines \tttf
\getbuffer
\stoplines

Again, this is consistent, so let's do some manipulation:

\startbuffer
\bgroup
         \setbox0\hbox{local}         \wd0=6em [\copy0:\the\wd0]
  \global\setbox0\hbox{global} \global\wd0=5em [\copy0:\the\wd0]
         \setbox0\hbox{local}         \wd0=6em [\copy0:\the\wd0]
\egroup
                                               [\copy0:\the\wd0]
\stopbuffer

\typebuffer

\startlines \tttf
\getbuffer
\stoplines

Right, no surprise here, but \unknown

\startbuffer
\bgroup
         \setbox0\hbox{local}  \wd0=6em [\copy0:\the\wd0]
  \global\setbox0\hbox{global} \wd0=5em [\copy0:\the\wd0]
         \setbox0\hbox{local}  \wd0=6em [\copy0:\the\wd0]
\egroup
                                        [\copy0:\the\wd0]
\stopbuffer

\typebuffer

See the difference? There is none. The second width assignment is applied to the
global box.

\startlines \tttf
\getbuffer
\stoplines

So how about this then:

\startbuffer
\bgroup
         \setbox0\hbox{local}  \wd0=6em [\copy0:\the\wd0]
  \global\setbox0\hbox{global}          [\copy0:\the\wd0]
         \setbox0\hbox{local}  \wd0=6em [\copy0:\the\wd0]
\egroup
                                        [\copy0:\the\wd0]
\stopbuffer

\typebuffer

Is this what you expect?

\startlines \tttf
\getbuffer
\stoplines

So, in the case of boxes, an assignment to a box dimension is applied to the last
instance of the register, and the global nature is kind of remembered. Inside a
group, registers that are accessed are pushed on a stack and the assignments are
applied to the one on the stack and when no local box is assigned, the one at the
outer level gets the treatment. You can also say that a global box is unreachable
once a local instance is used. \footnote {The code that implements \type
{\global\setbox} actually removes all intermediate boxes.}

\startbuffer
\setbox0\hbox{outer} [\copy0:\the\wd0]
\bgroup
            \wd0=6em [\copy0:\the\wd0]
\egroup
                     [\copy0:\the\wd0]
\stopbuffer

\typebuffer

This gives:

\startlines \tttf
\getbuffer
\stoplines

It works as expected when we use local boxes after such an assignment:

\startbuffer
\setbox0\hbox{outer}           [\copy0:\the\wd0]
\bgroup
                      \wd0=6em [\copy0:\the\wd0]
  \setbox0\hbox{inner (local)} [\copy0:\the\wd0]
\egroup
                               [\copy0:\the\wd0]
\stopbuffer

\typebuffer

This gives:

\startlines \tttf
\getbuffer
\stoplines

Interestingly in practice this is natural enough not to get noticed. Also, as the
\TEX book explicitly mentions that one should not mix local and global usage, not
many users will do that. For instance the scratch registers 0, 2, 4, 6 and 8 are
often used locally while 1, 3, 5, 7 and 9 are supposedly used global. The
argument for doing this is that it does not lead to unwanted stack build-up, but
the last examples given here provide another good reason. Actually, global
assignments happen seldom in macro packages, at least compared to local ones.

In \LUATEX\ we can also access boxes at the \LUA\ end. We can for instance change
the width as follows:

\startbuffer
\bgroup
         \setbox0\hbox{local}
         \ctxlua{tex.box[0].width = tex.sp("6em")} [\copy0:\the\wd0]
  \global\setbox0\hbox{global}
         \ctxlua{tex.box[0].width = tex.sp("5em")} [\copy0:\the\wd0]
         \setbox0\hbox{local}
         \ctxlua{tex.box[0].width = tex.sp("6em")} [\copy0:\the\wd0]
\egroup
                                                   [\copy0:\the\wd0]
\stopbuffer

\typebuffer

This is consistent with the \TEX\ end:

\startlines \tttf
\getbuffer
\stoplines

This is also true for:

\startbuffer
\bgroup
         \setbox0\hbox{local}
         \ctxlua{tex.box[0].width = tex.sp("6em")} [\copy0:\the\wd0]
  \global\setbox0\hbox{global}                     [\copy0:\the\wd0]
         \setbox0\hbox{local}
         \ctxlua{tex.box[0].width = tex.sp("6em")} [\copy0:\the\wd0]
\egroup
                                                   [\copy0:\the\wd0]
\stopbuffer

\typebuffer

Which gives:

\startlines \tttf
\getbuffer
\stoplines

The fact that a \type {\global} prefix is not needed for a global assignment at
the \TEX\ end means that we don't need a special function at the \LUA\ end for
assigning the width of a box. You won't miss it.

There is one catch when coding at the \TEX\ end. Imagine this:

\startbuffer
\setbox0\hbox{local} [\copy0:\the\wd0]
\bgroup
  \wd0=6em           [\copy0:\the\wd0]
\egroup
                     [\copy0:\the\wd0]
\stopbuffer

\typebuffer

In sync with what we told you will get:

\startlines \tttf
\getbuffer
\stoplines

However, this does not look that intuitive as the following:

\startbuffer
\setbox0\hbox{local} [\copy0:\the\wd0]
\bgroup
  \global\wd0=6em    [\copy0:\the\wd0]
\egroup
                     [\copy0:\the\wd0]
\stopbuffer

Here the global is redundant but it looks quite okay to put it there if only to
avoid confusion. \footnote {I finally decided to remove some of the \type
{\global} prefixes in my older code, but I must admit that I sometimes felt
reluctant when doing it, so I kept a few.}

\stopchapter

\stopcomponent
