% language=uk

% lua.newtable

\environment luametatex-style

\startcomponent luametatex-metapost

\startchapter[reference=metapost,title={The \METAPOST\ library \type {mplib}}]

\startsection[title={Process management}][library=mplib]

\topicindex {\METAPOST}
\topicindex {\METAPOST+mplib}
\topicindex {images+mplib}
\topicindex {images+\METAPOST}

\libindex{version}

The \METAPOST\ library interface registers itself in the table \type {mplib}. It
is based on \MPLIB\ version \ctxlua {context(mplib.version())}.

\subsection{\type {new}}

\libindex{new}

To create a new \METAPOST\ instance, call

\startfunctioncall
<mpinstance> mp = mplib.new({...})
\stopfunctioncall

This creates the \type {mp} instance object. The argument hash can have a number
of different fields, as follows:

\starttabulate[|l|l|pl|pl|]
\DB name               \BC type     \BC description              \BC default           \NC \NR
\TB
\NC \type{error_line}  \NC number   \NC error line width         \NC 79                \NC \NR
\NC \type{print_line}  \NC number   \NC line length in ps output \NC 100               \NC \NR
\NC \type{random_seed} \NC number   \NC the initial random seed  \NC variable          \NC \NR
\NC \type{math_mode}   \NC string   \NC the number system to use:
                                        \type {scaled},
                                        \type {double} or
                                      % \type {binary} or
                                        \type {decimal}          \NC \type {scaled}    \NC \NR
\NC \type{interaction} \NC string   \NC the interaction mode:
                                        \type {batch},
                                        \type {nonstop},
                                        \type {scroll} or
                                        \type {errorstop}        \NC \type {errorstop} \NC \NR
\NC \type{job_name}    \NC string   \NC a compatibility value    \NC \type {mpout}     \NC \NR
\NC \type{find_file}   \NC function \NC a function to find files \NC only local files  \NC \NR
\NC \type{utf8}        \NC boolean  \NC permit characters in the
                                        range 128 upto 255 to be
                                        part of names            \NC \type {false}     \NC \NR
\LL
\stoptabulate

The binary mode is no longer available in the \LUATEX\ version of \MPLIB. It
offers no real advantage and brings a ton of extra libraries with platform
specific properties that we can now avoid. We might introduce a high resolution
scaled variant at some point but only when it pays of performance wise.

The \type {find_file} function should be of this form:

\starttyping
<string> found = finder (<string> name, <string> mode, <string> type)
\stoptyping

with:

\starttabulate[|l|p|]
\DB name        \BC the requested file \NC \NR
\TB
\NC \type{mode} \NC the file mode: \type {r} or \type {w} \NC \NR
\NC \type{type} \NC the kind of file, one of: \type {mp}, \type {tfm}, \type {map},
                    \type {pfb}, \type {enc} \NC \NR
\LL
\stoptabulate

Return either the full path name of the found file, or \type {nil} if the file
cannot be found.

Note that the new version of \MPLIB\ no longer uses binary mem files, so the way
to preload a set of macros is simply to start off with an \type {input} command
in the first \type {execute} call.

When you are processing a snippet of text starting with \type {btex} and
ending with either \type {etex} or \type {verbatimtex}, the \METAPOST\
\type {texscriptmode} parameter controls how spaces and newlines get honoured.
The default value is~1. Possible values are:

\starttabulate[|l|p|]
\DB name      \BC meaning \NC \NR
\TB
\NC \type {0} \NC no newlines \NC \NR
\NC \type {1} \NC newlines in \type {verbatimtex} \NC \NR
\NC \type {2} \NC newlines in \type {verbatimtex} and \type {etex} \NC \NR
\NC \type {3} \NC no leading and trailing strip in \type {verbatimtex} \NC \NR
\NC \type {4} \NC no leading and trailing strip in \type {verbatimtex} and \type {btex} \NC \NR
\LL
\stoptabulate

That way the \LUA\ handler (assigned to \type {make_text}) can do what it likes.
An \type {etex} has to be followed by a space or \type {;} or be at the end of a
line and preceded by a space or at the beginning of a line.

\subsection{\type {statistics}}

\libindex{statistics}

You can request statistics with:

\startfunctioncall
<table> stats = mp:statistics()
\stopfunctioncall

This function returns the vital statistics for an \MPLIB\ instance. There are
four fields, giving the maximum number of used items in each of four allocated
object classes:

\starttabulate[|l|l|p|]
\DB field  \BC type \BC explanation \NC \NR
\TB
\NC \type{main_memory} \NC number \NC memory size \NC \NR
\NC \type{hash_size}   \NC number \NC hash size\NC \NR
\NC \type{param_size}  \NC number \NC simultaneous macro parameters\NC \NR
\NC \type{max_in_open} \NC number \NC input file nesting levels\NC \NR
\LL
\stoptabulate

Note that in the new version of \MPLIB, this is informational only. The objects
are all allocated dynamically, so there is no chance of running out of space
unless the available system memory is exhausted.

\subsection{\type {execute}}

\libindex{execute}

You can ask the \METAPOST\ interpreter to run a chunk of code by calling

\startfunctioncall
<table> rettable = execute(mp,"metapost code")
\stopfunctioncall

for various bits of \METAPOST\ language input. Be sure to check the \type
{rettable.status} (see below) because when a fatal \METAPOST\ error occurs the
\MPLIB\ instance will become unusable thereafter.

Generally speaking, it is best to keep your chunks small, but beware that all
chunks have to obey proper syntax, like each of them is a small file. For
instance, you cannot split a single statement over multiple chunks.

In contrast with the normal stand alone \type {mpost} command, there is
\notabene {no} implied \quote{input} at the start of the first chunk.

\subsection{\type {finish}}

\libindex{finish}

\startfunctioncall
<table> rettable = finish(mp)
\stopfunctioncall

If for some reason you want to stop using an \MPLIB\ instance while processing is
not yet actually done, you can call \type {finish}. Eventually, used memory
will be freed and open files will be closed by the \LUA\ garbage collector, but
an explicit \type {finish} is the only way to capture the final part of the
output streams.

\stopsection

\startsection[title={The end result}]

\libindex {fields}

The return value of \type {execute} and \type {finish} is a table with a
few possible keys (only \type {status} is always guaranteed to be present).

\starttabulate[|l|l|p|]
\DB field  \BC type \BC explanation \NC \NR
\TB
\NC \type{log}    \NC string \NC output to the \quote {log} stream \NC \NR
\NC \type{term}   \NC string \NC output to the \quote {term} stream \NC \NR
\NC \type{error}  \NC string \NC output to the \quote {error} stream
                                 (only used for \quote {out of memory}) \NC \NR
\NC \type{status} \NC number \NC the return value:
                                 \type {0} = good,
                                 \type {1} = warning,
                                 \type {2} = errors,
                                 \type {3} = fatal error \NC \NR
\NC \type{fig}    \NC table  \NC an array of generated figures (if any) \NC \NR
\LL
\stoptabulate

When \type {status} equals~3, you should stop using this \MPLIB\ instance
immediately, it is no longer capable of processing input.

If it is present, each of the entries in the \type {fig} array is a userdata
representing a figure object, and each of those has a number of object methods
you can call:

\starttabulate[|l|l|p|]
\DB field  \BC type \BC explanation \NC \NR
\TB
\NC \type{boundingbox}  \NC function \NC returns the bounding box, as an array of 4
                                         values \NC \NR
\NC \type{postscript}   \NC function \NC returns a string that is the ps output of the
                                         \type {fig}. this function accepts two optional
                                         integer arguments for specifying the values of
                                         \type {prologues} (first argument) and \type
                                         {procset} (second argument) \NC \NR
\NC \type{svg}          \NC function \NC returns a string that is the svg output of the
                                         \type {fig}. This function accepts an optional
                                         integer argument for specifying the value of
                                         \type {prologues} \NC \NR
\NC \type{objects}      \NC function \NC returns the actual array of graphic objects in
                                         this \type {fig} \NC \NR
\NC \type{copy_objects} \NC function \NC returns a deep copy of the array of graphic
                                         objects in this \type {fig} \NC \NR
\NC \type{filename}     \NC function \NC the filename this \type {fig}'s \POSTSCRIPT\
                                         output would have written to in stand alone
                                         mode \NC \NR
\NC \type{width}        \NC function \NC the \type {fontcharwd} value \NC \NR
\NC \type{height}       \NC function \NC the \type {fontcharht} value \NC \NR
\NC \type{depth}        \NC function \NC the \type {fontchardp} value \NC \NR
\NC \type{italcorr}     \NC function \NC the \type {fontcharit} value \NC \NR
\NC \type{charcode}     \NC function \NC the (rounded) \type {charcode} value \NC \NR
\LL
\stoptabulate

Note: you can call \type {fig:objects()} only once for any one \type {fig}
object!

When the boundingbox represents a \quote {negated rectangle}, i.e.\ when the
first set of coordinates is larger than the second set, the picture is empty.

Graphical objects come in various types that each has a different list of
accessible values. The types are: \type {fill}, \type {outline}, \type {text},
\type {start_clip}, \type {stop_clip}, \type {start_bounds}, \type {stop_bounds},
\type {special}.

There is a helper function (\type {mplib.fields(obj)}) to get the list of
accessible values for a particular object, but you can just as easily use the
tables given below.

All graphical objects have a field \type {type} that gives the object type as a
string value; it is not explicit mentioned in the following tables. In the
following, \type {number}s are \POSTSCRIPT\ points represented as a floating
point number, unless stated otherwise. Field values that are of type \type
{table} are explained in the next section.

\subsection{fill}

\starttabulate[|l|l|p|]
\DB field  \BC type \BC explanation \NC \NR
\TB
\NC \type{path}       \NC table  \NC the list of knots \NC \NR
\NC \type{htap}       \NC table  \NC the list of knots for the reversed trajectory \NC \NR
\NC \type{pen}        \NC table  \NC knots of the pen \NC \NR
\NC \type{color}      \NC table  \NC the object's color \NC \NR
\NC \type{linejoin}   \NC number \NC line join style (bare number)\NC \NR
\NC \type{miterlimit} \NC number \NC miterlimit\NC \NR
\NC \type{prescript}  \NC string \NC the prescript text \NC \NR
\NC \type{postscript} \NC string \NC the postscript text \NC \NR
\LL
\stoptabulate

The entries \type {htap} and \type {pen} are optional.

\subsection{outline}

\starttabulate[|l|l|p|]
\DB field  \BC type \BC explanation \NC \NR
\TB
\NC \type{path}       \NC table  \NC the list of knots \NC \NR
\NC \type{pen}        \NC table  \NC knots of the pen \NC \NR
\NC \type{color}      \NC table  \NC the object's color \NC \NR
\NC \type{linejoin}   \NC number \NC line join style (bare number) \NC \NR
\NC \type{miterlimit} \NC number \NC miterlimit \NC \NR
\NC \type{linecap}    \NC number \NC line cap style (bare number) \NC \NR
\NC \type{dash}       \NC table  \NC representation of a dash list \NC \NR
\NC \type{prescript}  \NC string \NC the prescript text \NC \NR
\NC \type{postscript} \NC string \NC the postscript text \NC \NR
\LL
\stoptabulate

The entry \type {dash} is optional.

\subsection{text}

\starttabulate[|l|l|p|]
\DB field  \BC type \BC explanation \NC \NR
\TB
\NC \type{text}       \NC string \NC the text \NC \NR
\NC \type{font}       \NC string \NC font tfm name \NC \NR
\NC \type{dsize}      \NC number \NC font size \NC \NR
\NC \type{color}      \NC table  \NC the object's color \NC \NR
\NC \type{width}      \NC number \NC \NC \NR
\NC \type{height}     \NC number \NC \NC \NR
\NC \type{depth}      \NC number \NC \NC \NR
\NC \type{transform}  \NC table  \NC a text transformation \NC \NR
\NC \type{prescript}  \NC string \NC the prescript text \NC \NR
\NC \type{postscript} \NC string \NC the postscript text \NC \NR
\LL
\stoptabulate

\subsection{special}

\starttabulate[|l|l|p|]
\DB field  \BC type \BC explanation \NC \NR
\TB
\NC \type{prescript} \NC string \NC special text \NC \NR
\LL
\stoptabulate

\subsection{start_bounds, start_clip}

\starttabulate[|l|l|p|]
\DB field  \BC type \BC explanation \NC \NR
\TB
\NC \type{path} \NC table \NC the list of knots \NC \NR
\LL
\stoptabulate

\subsubsection{stop_bounds, stop_clip}

Here are no fields available.

\stopsection

\startsection[title={Subsidiary table formats}]

\subsection{Paths and pens}

Paths and pens (that are really just a special type of paths as far as \MPLIB\ is
concerned) are represented by an array where each entry is a table that
represents a knot.

\starttabulate[|l|l|p|]
\DB field  \BC type \BC explanation \NC \NR
\TB
\NC \type{left_type}  \NC string \NC when present: endpoint, but usually absent \NC \NR
\NC \type{right_type} \NC string \NC like \type {left_type} \NC \NR
\NC \type{x_coord}    \NC number \NC X coordinate of this knot \NC \NR
\NC \type{y_coord}    \NC number \NC Y coordinate of this knot \NC \NR
\NC \type{left_x}     \NC number \NC X coordinate of the precontrol point of this knot \NC \NR
\NC \type{left_y}     \NC number \NC Y coordinate of the precontrol point of this knot \NC \NR
\NC \type{right_x}    \NC number \NC X coordinate of the postcontrol point of this knot \NC \NR
\NC \type{right_y}    \NC number \NC Y coordinate of the postcontrol point of this knot \NC \NR
\LL
\stoptabulate

There is one special case: pens that are (possibly transformed) ellipses have an
extra string-valued key \type {type} with value \type {elliptical} besides the
array part containing the knot list.

\subsection{Colors}

A color is an integer array with 0, 1, 3 or 4 values:

\starttabulate[|l|l|p|]
\DB field  \BC type \BC explanation \NC \NR
\TB
\NC \type{0} \NC marking only \NC no values                                                     \NC \NR
\NC \type{1} \NC greyscale    \NC one value in the range $(0,1)$, \quote {black} is $0$         \NC \NR
\NC \type{3} \NC \RGB         \NC three values in the range $(0,1)$, \quote {black} is $0,0,0$  \NC \NR
\NC \type{4} \NC \CMYK        \NC four values in the range $(0,1)$, \quote {black} is $0,0,0,1$ \NC \NR
\LL
\stoptabulate

If the color model of the internal object was \type {uninitialized}, then it was
initialized to the values representing \quote {black} in the colorspace \type
{defaultcolormodel} that was in effect at the time of the \type {shipout}.

\subsection{Transforms}

Each transform is a six|-|item array.

\starttabulate[|l|l|p|]
\DB index  \BC type \BC explanation \NC \NR
\TB
\NC \type{1} \NC number \NC represents x  \NC \NR
\NC \type{2} \NC number \NC represents y  \NC \NR
\NC \type{3} \NC number \NC represents xx \NC \NR
\NC \type{4} \NC number \NC represents yx \NC \NR
\NC \type{5} \NC number \NC represents xy \NC \NR
\NC \type{6} \NC number \NC represents yy \NC \NR
\LL
\stoptabulate

Note that the translation (index 1 and 2) comes first. This differs from the
ordering in \POSTSCRIPT, where the translation comes last.

\subsection{Dashes}

Each \type {dash} is two-item hash, using the same model as \POSTSCRIPT\ for the
representation of the dashlist. \type {dashes} is an array of \quote {on} and
\quote {off}, values, and \type {offset} is the phase of the pattern.

\starttabulate[|l|l|p|]
\DB field  \BC type \BC explanation \NC \NR
\TB
\NC \type{dashes} \NC hash   \NC an array of on-off numbers \NC \NR
\NC \type{offset} \NC number \NC the starting offset value  \NC \NR
\LL
\stoptabulate

\subsection{Pens and \type {pen_info}}

\libindex{pen_info}

There is helper function (\type {pen_info(obj)}) that returns a table containing
a bunch of vital characteristics of the used pen (all values are floats):

\starttabulate[|l|l|p|]
\DB field  \BC type \BC explanation \NC \NR
\TB
\NC \type{width} \NC number \NC width of the pen \NC \NR
\NC \type{sx}    \NC number \NC $x$ scale        \NC \NR
\NC \type{rx}    \NC number \NC $xy$ multiplier  \NC \NR
\NC \type{ry}    \NC number \NC $yx$ multiplier  \NC \NR
\NC \type{sy}    \NC number \NC $y$ scale        \NC \NR
\NC \type{tx}    \NC number \NC $x$ offset       \NC \NR
\NC \type{ty}    \NC number \NC $y$ offset       \NC \NR
\LL
\stoptabulate

\stopsection

\startsection[title=Acessors]

\subsection[title={Character size information}]

\libindex{char_width}
\libindex{char_height}
\libindex{char_depth}

These functions find the size of a glyph in a defined font. The \type {fontname}
is the same name as the argument to \type {infont}; the \type {char} is a glyph
id in the range 0 to 255; the returned \type {w} is in AFM units.


\startfunctioncall
<number> w = char_width(mp,<string> fontname, <number> char)
<number> h = char_height(mp,<string> fontname, <number> char)
<number> d = char_depth(mp,<string> fontname, <number> char)
\stopfunctioncall

These are rather useless and might become obsolete.

\subsection{\type {get_[boolean|numeric|string|path]}}

\libindex{get_boolean}
\libindex{get_numeric}
\libindex{get_path}
\libindex{get_string}

When a script call brings you from the \METAPOST\ run (temporarily) back to
\LUA\ you can access variables, but only if they are known (so for instance
anonymous capsules like loop variables are not accessible).

\startfunctioncall
<boolean> w = get_boolean(mp,<string> name)
<number>  n = get_numeric(mp,<string> name)
<string>  s = get_string (mp,<string> name)
<table>   p = get_path   (mp,<string> name)
\stopfunctioncall

The path is returned a a table with subtables that have six numbers: the
coordinates of the point, pre- and postcontrol. A \type {cycle} fields indicates
if a path is cyclic.

\stopsection

\stopchapter

\stopcomponent
