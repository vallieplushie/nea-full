\environment xml-mkiv-style

\startcomponent xml-mkiv-commands

\startchapter[title={Commands}]

\startsection[title={nodes and lpaths}]

The amount of commands available for manipulating the \XML\ file is rather large.
Many of the commands cooperate with the already discussed setups, a fancy name
for a collection of macro calls either or not mixed with text.

Most of the commands are just shortcuts to \LUA\ calls, which means that the real
work is done by \LUA. In fact, what happens is that we have a continuous transfer
of control from \TEX\ to \LUA, where \LUA\ prints back either data (like element
content or attribute values) or just invokes a setup whereby it passes a
reference to the node resolved conform the path expression. The invoked setup
itself might return control to \LUA\ again, etc.

This sounds complicated but examples will show what we mean here. First we
present the whole repertoire of commands. Because users can read the source code,
they might uncover more commands, but only the ones discussed here are official.
The commands are grouped in categories.

In the following sections \cmdinternal {cd:node} means a reference to a node:
this can be the identifier of the root (the loaded xml tree) or a reference to a
node in that tree (often the result of some lookup. A \cmdinternal {cd:lpath} is
a fancy name for a path expression (as with \XSLT) but resolved by \LUA.

\stopsection

\startsection[title={commands}]

There are a lot of commands available but you probably can ignore most of them.
We try to be complete which means that there is for instance \type {\xmlfirst} as
well as \type {\xmllast} but you probably never need the last one. There are also
commands that were used when testing this interface and we see no reason to
remove them. Some obscure ones are used in modules and after a while even I often
forget that they exist. To give you an idea of what commands are important we
show their use in generating the \CONTEXT\ command definitions (\type
{x-set-11.mkiv}) per January 2016:

\startcolumns[n=2,balance=yes]
\starttabulate[|l|r|]
\NC \type {\xmlall}                   \NC  1 \NC \NR
\NC \type {\xmlatt}                   \NC 23 \NC \NR
\NC \type {\xmlattribute}             \NC  1 \NC \NR
\NC \type {\xmlcount}                 \NC  1 \NC \NR
\NC \type {\xmldoif}                  \NC  2 \NC \NR
\NC \type {\xmldoifelse}              \NC  1 \NC \NR
\NC \type {\xmlfilterlist}            \NC  4 \NC \NR
\NC \type {\xmlflush}                 \NC  5 \NC \NR
\NC \type {\xmlinclude}               \NC  1 \NC \NR
\NC \type {\xmlloadonly}              \NC  1 \NC \NR
\NC \type {\xmlregisterdocumentsetup} \NC  1 \NC \NR
\NC \type {\xmlsetsetup}              \NC  1 \NC \NR
\NC \type {\xmlsetup}                 \NC  4 \NC \NR
\stoptabulate
\stopcolumns

As you can see filtering, flushing and accessing attributes score high. Below we show
the statistics of a quite complex rendering (5 variants of schoolbooks: basic book,
answers, teachers guide, worksheets, full blown version with extensive tracing).

\startcolumns[n=2,balance=yes]
\starttabulate[|l|r|]
\NC \type {\xmladdindex}              \NC   3 \NC \NR
\NC \type {\xmlall}                   \NC   5 \NC \NR
\NC \type {\xmlappendsetup}           \NC   1 \NC \NR
\NC \type {\xmlapplyselectors}        \NC   1 \NC \NR
\NC \type {\xmlatt}                   \NC  40 \NC \NR
\NC \type {\xmlattdef}                \NC   9 \NC \NR
\NC \type {\xmlattribute}             \NC  10 \NC \NR
\NC \type {\xmlbadinclusions}         \NC   3 \NC \NR
\NC \type {\xmlconcat}                \NC   3 \NC \NR
\NC \type {\xmlcount}                 \NC   1 \NC \NR
\NC \type {\xmldelete}                \NC  11 \NC \NR
\NC \type {\xmldoif}                  \NC  39 \NC \NR
\NC \type {\xmldoifelse}              \NC  28 \NC \NR
\NC \type {\xmldoifelsetext}          \NC  13 \NC \NR
\NC \type {\xmldoifnot}               \NC   2 \NC \NR
\NC \type {\xmldoifnotselfempty}      \NC   1 \NC \NR
\NC \type {\xmlfilter}                \NC 100 \NC \NR
\NC \type {\xmlfirst}                 \NC  51 \NC \NR
\NC \type {\xmlflush}                 \NC  69 \NC \NR
\NC \type {\xmlflushcontext}          \NC   2 \NC \NR
\NC \type {\xmlinclude}               \NC   1 \NC \NR
\NC \type {\xmlincludeoptions}        \NC   5 \NC \NR
\NC \type {\xmlinclusion}             \NC  16 \NC \NR
\NC \type {\xmlinjector}              \NC   1 \NC \NR
\NC \type {\xmlloaddirectives}        \NC   1 \NC \NR
\NC \type {\xmlmapvalue}              \NC   4 \NC \NR
\NC \type {\xmlmatch}                 \NC   1 \NC \NR
\NC \type {\xmlprependsetup}          \NC   5 \NC \NR
\NC \type {\xmlregisterdocumentsetup} \NC   2 \NC \NR
\NC \type {\xmlregistersetup}         \NC   1 \NC \NR
\NC \type {\xmlremapnamespace}        \NC   1 \NC \NR
\NC \type {\xmlsetfunction}           \NC   2 \NC \NR
\NC \type {\xmlsetinjectors}          \NC   2 \NC \NR
\NC \type {\xmlsetsetup}              \NC  11 \NC \NR
\NC \type {\xmlsetup}                 \NC  76 \NC \NR
\NC \type {\xmlstrip}                 \NC   1 \NC \NR
\NC \type {\xmlstripanywhere}         \NC   1 \NC \NR
\NC \type {\xmltag}                   \NC   1 \NC \NR
\NC \type {\xmltext}                  \NC  53 \NC \NR
\NC \type {\xmlvalue}                 \NC   2 \NC \NR
\stoptabulate
\stopcolumns

Here many more are used but this is an exceptional case. The top is again
dominated by filtering, flushing and attribute consulting. The list can actually
be smaller. For instance, the \type {\xmlcount} can just as well be \type
{\xmlfilter} with a \type {count} finalizer. There are also some special ones,
like the injectors, that are needed for finetuning the final result.

\stopsection

\startsection[title={loading}]

\startxmlcmd {\cmdbasicsetup{xmlloadfile}}
    loads the file \cmdinternal {cd:file} and registers it under \cmdinternal
    {cd:name} and applies either given or standard \cmdinternal
    {cd:xmlsetup} (alias: \type {\xmlload})
\stopxmlcmd

\startxmlcmd {\cmdbasicsetup{xmlloadbuffer}}
    loads the buffer \cmdinternal {cd:buffer} and registers it under
    \cmdinternal {cd:name} and applies either given or standard
    \cmdinternal {cd:xmlsetup}
\stopxmlcmd

\startxmlcmd {\cmdbasicsetup{xmlloaddata}}
    loads \cmdinternal {cd:text} and registers it under \cmdinternal
    {cd:name} and applies either given or standard \cmdinternal
    {cd:xmlsetup}
\stopxmlcmd

\startxmlcmd {\cmdbasicsetup{xmlloadonly}}
    loads \cmdinternal {cd:text} and registers it under \cmdinternal
    {cd:name} and applies either given or standard \cmdinternal
    {cd:xmlsetup} but doesn't flush the content
\stopxmlcmd

\startxmlcmd {\cmdbasicsetup{xmlinclude}}
    includes the file specified by attribute \cmdinternal {cd:name} of the
    element located by \cmdinternal {cd:lpath} at node \cmdinternal {cd:node}
\stopxmlcmd

\startxmlcmd {\cmdbasicsetup{xmlprocessfile}}
    registers file \cmdinternal {cd:file} as \cmdinternal {cd:name} and
    process the tree starting with \cmdinternal {cd:xmlsetup} (alias:
    \type {\xmlprocess})
\stopxmlcmd

\startxmlcmd {\cmdbasicsetup{xmlprocessbuffer}}
    registers buffer \cmdinternal {cd:name} as \cmdinternal {cd:name} and process
    the tree starting with \cmdinternal {cd:xmlsetup}
\stopxmlcmd

\startxmlcmd {\cmdbasicsetup{xmlprocessdata}}
    registers \cmdinternal {cd:text} as \cmdinternal {cd:name} and process
    the tree starting with \cmdinternal {cd:xmlsetup}
\stopxmlcmd

The initial setup defaults to \type {xml:process} that is defined
as follows:

\starttyping
\startsetups xml:process
  \xmlregistereddocumentsetups\xmldocument
  \xmlmain\xmldocument
\stopsetups
\stoptyping

First we apply the setups associated with the document (including common setups)
and then we flush the whole document. The macro \type {\xmldocument} expands to
the current document id. There is also \type {\xmlself} which expands to the
current node number (\type {#1} in setups).

\startxmlcmd {\cmdbasicsetup{xmlmain}}
    returns the whole document
\stopxmlcmd

Normally such a flush will trigger a chain reaction of setups associated with the
child elements.

\stopsection

\startsection[title={saving}]

\startxmlcmd {\cmdbasicsetup{xmlsave}}
    saves the given node \cmdinternal {cd:node} in the file \cmdinternal {cd:file}
\stopxmlcmd

\startxmlcmd {\cmdbasicsetup{xmltofile}}
    saves the match of \cmdinternal {cd:lpath} in the file \cmdinternal {cd:file}
\stopxmlcmd

\startxmlcmd {\cmdbasicsetup{xmltobuffer}}
    saves the match of \cmdinternal {cd:lpath} in the buffer \cmdinternal {cd:buffer}
\stopxmlcmd

\startxmlcmd {\cmdbasicsetup{xmltobufferverbose}}
    saves the match of \cmdinternal {cd:lpath} verbatim in the buffer \cmdinternal
    {cd:buffer}
\stopxmlcmd

% \startxmlcmd {\cmdbasicsetup{xmltoparameters}}
%     converts the match of \cmdinternal {cd:lpath} to key|/|values (for tracing)
% \stopxmlcmd

The next command is only needed when you have messed with the tree using
\LUA\ code.

\startxmlcmd {\cmdbasicsetup{xmladdindex}}
    (re)indexes a tree
\stopxmlcmd

The following macros are only used in special situations and are not really meant
for users.

\startxmlcmd {\cmdbasicsetup{xmlraw}}
    flush the content if \cmdinternal {cd:node} with original entities
\stopxmlcmd

\startxmlcmd {\cmdbasicsetup{startxmlraw}}
    flush the wrapped content with original entities
\stopxmlcmd

\stopsection

\startsection[title={flushing data}]

When we flush an element, the associated \XML\ setups are expanded. The most
straightforward way to flush an element is the following. Keep in mind that the
returned values itself can trigger setups and therefore flushes.

\startxmlcmd {\cmdbasicsetup{xmlflush}}
    returns all nodes under \cmdinternal {cd:node}
\stopxmlcmd

You can restrict flushing by using commands that accept a specification.

\startxmlcmd {\cmdbasicsetup{xmltext}}
    returns the text of the matching \cmdinternal {cd:lpath} under \cmdinternal
    {cd:node}
\stopxmlcmd

\startxmlcmd {\cmdbasicsetup{xmlpure}}
    returns the text of the matching \cmdinternal {cd:lpath} under \cmdinternal
    {cd:node} without \type {\Ux} escaped special \TEX\ characters
\stopxmlcmd

\startxmlcmd {\cmdbasicsetup{xmlflushtext}}
    returns the text of the \cmdinternal {cd:node}
\stopxmlcmd

\startxmlcmd {\cmdbasicsetup{xmlflushpure}}
    returns the text of the \cmdinternal {cd:node} without \type {\Ux} escaped
    special \TEX\ characters
\stopxmlcmd

\startxmlcmd {\cmdbasicsetup{xmlnonspace}}
    returns the text of the matching \cmdinternal {cd:lpath} under \cmdinternal
    {cd:node} without embedded spaces
\stopxmlcmd

\startxmlcmd {\cmdbasicsetup{xmlall}}
    returns all nodes under \cmdinternal {cd:node} that matches \cmdinternal
    {cd:lpath}
\stopxmlcmd

\startxmlcmd {\cmdbasicsetup{xmllastmatch}}
    returns all nodes found in the last match
\stopxmlcmd

\startxmlcmd {\cmdbasicsetup{xmlfirst}}
    returns the first node under \cmdinternal {cd:node} that matches \cmdinternal
    {cd:lpath}
\stopxmlcmd

\startxmlcmd {\cmdbasicsetup{xmllast}}
    returns the last node under \cmdinternal {cd:node} that matches \cmdinternal
    {cd:lpath}
\stopxmlcmd

\startxmlcmd {\cmdbasicsetup{xmlfilter}}
    at a match of \cmdinternal {cd:lpath} a given filter \type {filter} is applied
    and the result is returned
\stopxmlcmd

\startxmlcmd {\cmdbasicsetup{xmlsnippet}}
    returns the \cmdinternal {cd:number}\high{th} element under \cmdinternal
    {cd:node}
\stopxmlcmd

\startxmlcmd {\cmdbasicsetup{xmlposition}}
    returns the \cmdinternal {cd:number}\high{th} match of \cmdinternal
    {cd:lpath} at node \cmdinternal {cd:node}; a negative number starts at the
    end (alias: \type {\xmlindex})
\stopxmlcmd

\startxmlcmd {\cmdbasicsetup{xmlelement}}
    returns the \cmdinternal {cd:number}\high{th} child of node \cmdinternal {cd:node};
    a negative number starts at the end
\stopxmlcmd

\startxmlcmd {\cmdbasicsetup{xmlpos}}
    returns the index (position) in the parent node of \cmdinternal {cd:node}
\stopxmlcmd

\startxmlcmd {\cmdbasicsetup{xmlconcat}}
    returns the sequence of nodes that match \cmdinternal {cd:lpath} at
    \cmdinternal {cd:node} whereby \cmdinternal {cd:text} is put between each
    match
\stopxmlcmd

\startxmlcmd {\cmdbasicsetup{xmlconcatrange}}
    returns the \cmdinternal {cd:first}\high {th} upto \cmdinternal
    {cd:last}\high {th} of nodes that match \cmdinternal {cd:lpath} at
    \cmdinternal {cd:node} whereby \cmdinternal {cd:text} is put between each
    match
\stopxmlcmd

\startxmlcmd {\cmdbasicsetup{xmlcommand}}
    apply the given \cmdinternal {cd:xmlsetup} to each match of \cmdinternal
    {cd:lpath} at node \cmdinternal {cd:node}
\stopxmlcmd

\startxmlcmd {\cmdbasicsetup{xmlstrip}}
    remove leading and trailing spaces from nodes under \cmdinternal {cd:node}
    that match \cmdinternal {cd:lpath}
\stopxmlcmd

\startxmlcmd {\cmdbasicsetup{xmlstripped}}
    remove leading and trailing spaces from nodes under \cmdinternal {cd:node}
    that match \cmdinternal {cd:lpath} and return the content afterwards
\stopxmlcmd

\startxmlcmd {\cmdbasicsetup{xmlstripnolines}}
    remove leading and trailing spaces as well as collapse embedded spaces
    from nodes under \cmdinternal {cd:node} that match \cmdinternal {cd:lpath}
\stopxmlcmd

\startxmlcmd {\cmdbasicsetup{xmlstrippednolines}}
    remove leading and trailing spaces as well as collapse embedded spaces from
    nodes under \cmdinternal {cd:node} that match \cmdinternal {cd:lpath} and
    return the content afterwards
\stopxmlcmd

\startxmlcmd {\cmdbasicsetup{xmlverbatim}}
    flushes the content verbatim code (without any wrapping, i.e. no fonts
    are selected and such)
\stopxmlcmd

\startxmlcmd {\cmdbasicsetup{xmlinlineverbatim}}
    return the content of the node as inline verbatim code; no further
    interpretation (expansion) takes place and spaces are honoured; it uses the
    following wrapper
\stopxmlcmd

\startxmlcmd {\cmdbasicsetup{startxmlinlineverbatim}}
    wraps inline verbatim mode using the environment specified (a prefix \type
    {xml:} is added to the environment name)
\stopxmlcmd

\startxmlcmd {\cmdbasicsetup{xmldisplayverbatim}}
    return the content of the node as display verbatim code; no further
    interpretation (expansion) takes place and leading and trailing spaces and
    newlines are treated special; it uses the following wrapper
\stopxmlcmd

\startxmlcmd {\cmdbasicsetup{startxmldisplayverbatim}}
    wraps the content in display verbatim using the environment specified (a prefix
    \type {xml:} is added to the environment name)
\stopxmlcmd

\startxmlcmd {\cmdbasicsetup{xmlprettyprint}}
    pretty print (with colors) the node \cmdinternal {cd:node}; use the \CONTEXT\
    \SCITE\ lexers when available (\type {\usemodule [scite]})
\stopxmlcmd

\startxmlcmd {\cmdbasicsetup{xmlflushspacewise}}
    flush node \cmdinternal {cd:node} obeying spaces and newlines
\stopxmlcmd

\startxmlcmd {\cmdbasicsetup{xmlflushlinewise}}
    flush node \cmdinternal {cd:node} obeying newlines
\stopxmlcmd

\stopsection

\startsection[title={information}]

The following commands return strings. Normally these are used in tests.

\startxmlcmd {\cmdbasicsetup{xmlname}}
    returns the complete name (including namespace prefix) of the
    given \cmdinternal {cd:node}
\stopxmlcmd

\startxmlcmd {\cmdbasicsetup{xmlnamespace}}
    returns the namespace of the given \cmdinternal {cd:node}
\stopxmlcmd

\startxmlcmd {\cmdbasicsetup{xmltag}}
    returns the tag of the element, without namespace prefix
\stopxmlcmd

\startxmlcmd {\cmdbasicsetup{xmlcount}}
    returns the number of matches of \cmdinternal {cd:lpath} at node \cmdinternal
    {cd:node}
\stopxmlcmd

\startxmlcmd {\cmdbasicsetup{xmlatt}}
    returns the value of attribute \cmdinternal {cd:name} or empty if no such
    attribute exists
\stopxmlcmd

\startxmlcmd {\cmdbasicsetup{xmlattdef}}
    returns the value of attribute \cmdinternal {cd:name} or \cmdinternal
    {cd:string} if no such attribute exists
\stopxmlcmd

\startxmlcmd {\cmdbasicsetup{xmlrefatt}}
    returns the value of attribute \cmdinternal {cd:name} or empty if no such
    attribute exists; a leading \type {#} is removed (nicer for tex)
\stopxmlcmd

\startxmlcmd {\cmdbasicsetup{xmlchainatt}}
    returns the value of attribute \cmdinternal {cd:name} or empty if no such
    attribute exists; backtracks till a match is found
\stopxmlcmd

\startxmlcmd {\cmdbasicsetup{xmlchainattdef}}
    returns the value of attribute \cmdinternal {cd:name} or \cmdinternal
    {cd:string} if no such attribute exists; backtracks till a match is found
\stopxmlcmd

\startxmlcmd {\cmdbasicsetup{xmlattribute}}
    finds a first match for \cmdinternal {cd:lpath} at \cmdinternal {cd:node} and
    returns the value of attribute \cmdinternal {cd:name} or empty if no such
    attribute exists
\stopxmlcmd

\startxmlcmd {\cmdbasicsetup{xmlattributedef}}
    finds a first match for \cmdinternal {cd:lpath} at \cmdinternal {cd:node} and
    returns the value of attribute \cmdinternal {cd:name} or \cmdinternal
    {cd:text} if no such attribute exists
\stopxmlcmd

\startxmlcmd {\cmdbasicsetup{xmllastatt}}
    returns the last attribute found (this avoids a lookup)
\stopxmlcmd

\startxmlcmd {\cmdbasicsetup{xmlsetatt}}
    set the value of attribute \cmdinternal {cd:name}
\stopxmlcmd

\startxmlcmd {\cmdbasicsetup{xmlsetattribute}}
    set the value of attribute \cmdinternal {cd:name} for each match of \cmdinternal
    {cd:lpath}
\stopxmlcmd

\stopsection

\startsection[title={manipulation}]

You can use \LUA\ code to manipulate the tree and it makes no sense to duplicate
this in \TEX. In the future we might provide an interface to some of this
functionality. Keep in mind that manipuating the tree might have side effects as
we maintain several indices into the tree that also needs to be updated then.

\stopsection

\startsection[title={integration}]

If you write a module that deals with \XML, for instance processing cals tables,
then you need ways to control specific behaviour. For instance, you might want to
add a background to the table. Such directives are collected in \XML\ files and
can be loaded on demand.

\startxmlcmd {\cmdbasicsetup{xmlloaddirectives}}
    loads \CONTEXT\ directives from \cmdinternal {cd:file} that will get
    interpreted when processing documents
\stopxmlcmd

A directives definition file looks as follows:

\starttyping
<?xml version="1.0" standalone="yes"?>

<directives>
  <directive attribute='id' value="100"
    setup="cdx:100"/>
  <directive attribute='id' value="101"
    setup="cdx:101"/>
  <directive attribute='cdx' value="colors" element="cals:table"
    setup="cdx:cals:table:colors"/>
  <directive attribute='cdx' value="vertical" element="cals:table"
    setup="cdx:cals:table:vertical"/>
  <directive attribute='cdx' value="noframe" element="cals:table"
    setup="cdx:cals:table:noframe"/>
  <directive attribute='cdx' value="*" element="cals:table"
    setup="cdx:cals:table:*"/>
</directives>
\stoptyping

Examples of usage can be found in \type {x-cals.mkiv}. The directive is triggered
by an attribute. Instead of a setup you can specify a setup to be applied before
and after the node gets flushed.

\startxmlcmd {\cmdbasicsetup{xmldirectives}}
    apply the setups directive associated with the node
\stopxmlcmd

\startxmlcmd {\cmdbasicsetup{xmldirectivesbefore}}
    apply the before directives associated with the node
\stopxmlcmd

\startxmlcmd {\cmdbasicsetup{xmldirectivesafter}}
    apply the after directives associated with the node
\stopxmlcmd

\startxmlcmd {\cmdbasicsetup{xmlinstalldirective}}
    defines a directive that hooks into a handler
\stopxmlcmd

Normally a directive will be put in the \XML\ file, for instance as:

\starttyping
<?context-mathml-directive minus reduction yes ?>
\stoptyping

Here the \type {mathml} is the general class of directives and \type {minus} a
subclass, in our case a specific element.

\stopsection

\startsection[title={setups}]

The basic building blocks of \XML\ processing are setups. These are just
collections of macros that are expanded. These setups get one argument passed
(\type {#1}):

\starttyping
\startxmlsetups somedoc:somesetup
    \xmlflush{#1}
\stopxmlsetups
\stoptyping

This argument is normally a number that internally refers to a specific node in
the \XML\ tree. The user should see it as an abstract reference and not depend on
its numeric property. Just think of it as \quote {the current node}. You can (and
probably will) call such setups using:

\startxmlcmd {\cmdbasicsetup{xmlsetup}}
    expands setup \cmdinternal {cd:setup} and pass \cmdinternal {cd:node} as
    argument
\stopxmlcmd

However, in most cases the setups are associated to specific elements,
something that users of \XSLT\ might recognize as templates.

\startxmlcmd {\cmdbasicsetup{xmlsetfunction}}
    associates function \cmdinternal {cd:luafunction} to the elements in
    namespace \cmdinternal {cd:name} that match \cmdinternal {cd:lpath}
\stopxmlcmd

\startxmlcmd {\cmdbasicsetup{xmlsetsetup}}
    associates setups \cmdinternal {cd:setup} (\TEX\ code) with the matching
    nodes of \cmdinternal {cd:lpath} or root \cmdinternal {cd:node}
\stopxmlcmd

\startxmlcmd {\cmdbasicsetup{xmlprependsetup}}
    pushes \cmdinternal {cd:setup} to the front of global list of setups
\stopxmlcmd

\startxmlcmd {\cmdbasicsetup{xmlappendsetup}}
    adds \cmdinternal {cd:setup} to the global list of setups to be applied
    (alias: \type{\xmlregistersetup})
\stopxmlcmd

\startxmlcmd {\cmdbasicsetup{xmlbeforesetup}}
    pushes \cmdinternal {cd:setup} into the global list of setups; the
    last setup is the position
\stopxmlcmd

\startxmlcmd {\cmdbasicsetup{xmlaftersetup}}
    adds \cmdinternal {cd:setup} to the global list of setups; the last setup
    is the position
\stopxmlcmd

\startxmlcmd {\cmdbasicsetup{xmlremovesetup}}
    removes \cmdinternal {cd:setup} from the global list of setups
\stopxmlcmd

\startxmlcmd {\cmdbasicsetup{xmlprependdocumentsetup}}
    pushes \cmdinternal {cd:setup} to the front of list of setups to be applied
    to \cmdinternal {cd:name}
\stopxmlcmd

\startxmlcmd {\cmdbasicsetup{xmlappenddocumentsetup}}
    adds \cmdinternal {cd:setup} to the list of setups to be applied to
    \cmdinternal {cd:name} (you can also use the alias: \type
    {\xmlregisterdocumentsetup})
\stopxmlcmd

\startxmlcmd {\cmdbasicsetup{xmlbeforedocumentsetup}}
    pushes \cmdinternal {cd:setup} into the setups to be applied to \cmdinternal
    {cd:name}; the last setup is the position
\stopxmlcmd

\startxmlcmd {\cmdbasicsetup{xmlafterdocumentsetup}}
    adds \cmdinternal {cd:setup} to the setups to be applied to \cmdinternal
    {cd:name}; the last setup is the position
\stopxmlcmd

\startxmlcmd {\cmdbasicsetup{xmlremovedocumentsetup}}
    removes \cmdinternal {cd:setup} from the global list of setups to be applied
    to \cmdinternal {cd:name}
\stopxmlcmd

\startxmlcmd {\cmdbasicsetup{xmlresetsetups}}
    removes all global setups
\stopxmlcmd

\startxmlcmd {\cmdbasicsetup{xmlresetdocumentsetups}}
    removes all setups from the \cmdinternal {cd:name} specific list of setups to
    be applied
\stopxmlcmd

\startxmlcmd {\cmdbasicsetup{xmlflushdocumentsetups}{setup}}
    applies \cmdinternal {cd:setup} (can be a list) to \cmdinternal {cd:name}
\stopxmlcmd

\startxmlcmd {\cmdbasicsetup{xmlregisteredsetups}}
    applies all global setups to the current document
\stopxmlcmd

\startxmlcmd {\cmdbasicsetup{xmlregistereddocumentsetups}}
    applies all document specific \cmdinternal {cd:setup} to document
    \cmdinternal {cd:name}
\stopxmlcmd

\stopsection

\startsection[title={testing}]

The following test macros all take a \cmdinternal {cd:node} as first argument
and an \cmdinternal {cd:lpath} as second:

\startxmlcmd {\cmdbasicsetup{xmldoif}}
    expands to \cmdinternal {cd:true} when \cmdinternal {cd:lpath} matches at
    node \cmdinternal {cd:node}
\stopxmlcmd

\startxmlcmd {\cmdbasicsetup{xmldoifnot}}
    expands to \cmdinternal {cd:true} when \cmdinternal {cd:lpath} does not match
    at node \cmdinternal {cd:node}
\stopxmlcmd

\startxmlcmd {\cmdbasicsetup{xmldoifelse}}
    expands to \cmdinternal {cd:true} when \cmdinternal {cd:lpath} matches at
    node \cmdinternal {cd:node} and to \cmdinternal {cd:false} otherwise
\stopxmlcmd

\startxmlcmd {\cmdbasicsetup{xmldoiftext}}
    expands to \cmdinternal {cd:true} when the node matching \cmdinternal
    {cd:lpath} at node \cmdinternal {cd:node} has some content
\stopxmlcmd

\startxmlcmd {\cmdbasicsetup{xmldoifnottext}}
    expands to \cmdinternal {cd:true} when the node matching \cmdinternal
    {cd:lpath} at node \cmdinternal {cd:node} has no content
\stopxmlcmd

\startxmlcmd {\cmdbasicsetup{xmldoifelsetext}}
    expands to \cmdinternal {cd:true} when the node matching \cmdinternal
    {cd:lpath} at node \cmdinternal {cd:node} has content and to \cmdinternal
    {cd:false} otherwise
\stopxmlcmd

\startxmlcmd {\cmdbasicsetup{xmldoifatt}}
    expands to \cmdinternal {cd:true} when the attribute matching \cmdinternal
    {cd:node} and the name given as second argument matches the third argument
\stopxmlcmd

\startxmlcmd {\cmdbasicsetup{xmldoifnotatt}}
    expands to \cmdinternal {cd:true} when the attribute matching \cmdinternal
    {cd:node} and the name given as second argument differs from the third
    argument
\stopxmlcmd

\startxmlcmd {\cmdbasicsetup{xmldoifelseatt}}
    expands to \cmdinternal {cd:true} when the attribute matching \cmdinternal
    {cd:node} and the name given as second argument matches the third argument
    and to \cmdinternal {cd:false} otherwise
\stopxmlcmd

\startxmlcmd {\cmdbasicsetup{xmldoifelseempty}}
    expands to \cmdinternal {cd:true} when the node matching \cmdinternal
    {cd:lpath} at node \cmdinternal {cd:node} is empty and to \cmdinternal
    {cd:false} otherwise
\stopxmlcmd

\startxmlcmd {\cmdbasicsetup{xmldoifelseselfempty}}
    expands to \cmdinternal {cd:true} when the node is empty and to \cmdinternal
    {cd:false} otherwise
\stopxmlcmd

\startxmlcmd {\cmdbasicsetup{xmldoifselfempty}}
    expands to \cmdinternal {cd:true} when \cmdinternal {cd:node} is empty
\stopxmlcmd

\startxmlcmd {\cmdbasicsetup{xmldoifnotselfempty}}
    expands to \cmdinternal {cd:true} when \cmdinternal {cd:node} is not empty
\stopxmlcmd

\stopsection

\startsection[title={initialization}]

The general setup command (not to be confused with setups) that deals with the
\MKIV\ tree handler is \type {\setupxml}. There are currently only a few options.

\cmdfullsetup{setupxml}

When you set \type {default} to \cmdinternal {cd:text} elements with no setup
assigned will end up as text. When set to \type {hidden} such elements will be
hidden. You can apply the default yourself using:

\startxmlcmd {\cmdbasicsetup{xmldefaulttotext}}
    presets the tree with root \cmdinternal {cd:node} to the handlers set up with
    \type {\setupxml} option \cmdinternal{default}
\stopxmlcmd

You can set \type {compress} to \type {yes} in which case comment is stripped
from the tree when the file is read.

\startxmlcmd {\cmdbasicsetup{xmlregisterns}}
    associates an internal namespace (like \type {mml}) with one given in the
    document as \URL\ (like mathml)
\stopxmlcmd

\startxmlcmd {\cmdbasicsetup{xmlremapname}}
    changes the namespace and tag of the matching elements
\stopxmlcmd

\startxmlcmd {\cmdbasicsetup{xmlremapnamespace}}
    replaces all references to the given namespace to a new one (applied
    recursively)
\stopxmlcmd

\startxmlcmd {\cmdbasicsetup{xmlchecknamespace}}
    sets the namespace of the matching elements unless a namespace is already set
\stopxmlcmd

\stopsection

\startsection[title={helpers}]

Often an attribute will determine the rendering and this may result in many
tests. Especially when we have multiple attributes that control the output such
tests can become rather extensive and redundant because one gets $n\times m$ or
more such tests.

Therefore we have a convenient way to map attributes onto for instance strings or
commands.

\startxmlcmd {\cmdbasicsetup{xmlmapvalue}}
    associate a \cmdinternal {cd:text} with a \cmdinternal {cd:category} and
    \cmdinternal {cd:name} (alias: \type{\xmlmapval})
\stopxmlcmd

\startxmlcmd {\cmdbasicsetup{xmlvalue}}
    expand the value associated with a \cmdinternal {cd:category} and
    \cmdinternal {cd:name} and if not resolved, expand to the \cmdinternal
    {cd:text} (alias: \type{\xmlval})
\stopxmlcmd

\startxmlcmd {\cmdbasicsetup{xmldoifelsevalue}}
    associate a \cmdinternal {cd:text} with a \cmdinternal {cd:category} and
    \cmdinternal {cd:name}
\stopxmlcmd

This is used as follows. We define a couple of mappings in the same category:

\starttyping
\xmlmapvalue{emph}{bold}  {\bf}
\xmlmapvalue{emph}{italic}{\it}
\stoptyping

Assuming that we have associated the following setup with the \type {emph}
element, we can say (with \type {#1} being the current element):

\starttyping
\startxmlsetups demo:emph
  \begingroup
    \xmlvalue{emph}{\xmlatt{#1}{type}}{}
  \endgroup
\stopxmlsetups
\stoptyping

In this case we have no default. The \type {type} attribute triggers the actions,
as in:

\starttyping
normal <emph type='bold'>bold</emph> normal
\stoptyping

This mechanism is not really bound to elements and attributes so you can use this
mechanism for other purposes as well.

\stopsection

\startsection[title={Parameters}]

\startbuffer[test]
<something whatever="alpha">
    <what>
        beta
    </what>
</something>
\stopbuffer

\startbuffer
\startxmlsetups xml:mysetups
    \xmlsetsetup{\xmldocument}{*}{xml:*}
\stopxmlsetups

\xmlregistersetup{xml:mysetups}

\startxmlsetups xml:something
    parameter : \xmlpar   {#1}{whatever}\par
    attribute : \xmlatt   {#1}{whatever}\par
    text      : \xmlfirst {#1}{what}    \par
                \xmlsetpar{#1}{whatever}{gamma}
    parameter : \xmlpar   {#1}{whatever}\par
    \xmlflush{#1}
\stopxmlsetups

\startxmlsetups xml:what
    what: \xmlflush{#1}\par
    parameter : \xmlparam{#1}{..}{whatever}\par
\stopxmlsetups

\xmlprocessbuffer{main}{test}{}
\stopbuffer

Say that we have this \XML\ blob:

\typebuffer[test]

With:

\typebuffer

we get:

\getbuffer

Parameters are stored with a node.

\startxmlcmd {\cmdbasicsetup{xmlpar}}
    returns the value of parameter \cmdinternal {cd:name} or empty if no such
    parameter exists
\stopxmlcmd

\startxmlcmd {\cmdbasicsetup{xmlparam}}
    finds a first match for \cmdinternal {cd:lpath} at \cmdinternal {cd:node} and
    returns the value of parameter \cmdinternal {cd:name} or empty if no such
    parameter exists
\stopxmlcmd

\startxmlcmd {\cmdbasicsetup{xmllastpar}}
    returns the last parameter found (this avoids a lookup)
\stopxmlcmd

\startxmlcmd {\cmdbasicsetup{xmlsetpar}}
    set the value of parameter \cmdinternal {cd:name}
\stopxmlcmd

\startxmlcmd {\cmdbasicsetup{xmlsetparam}}
    set the value of parameter \cmdinternal {cd:name} for each match of \cmdinternal
    {cd:lpath}
\stopxmlcmd

\stopsection

\stopchapter

\stopcomponent
